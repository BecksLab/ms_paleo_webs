% Options for packages loaded elsewhere
% Options for packages loaded elsewhere
\PassOptionsToPackage{unicode}{hyperref}
\PassOptionsToPackage{hyphens}{url}
\PassOptionsToPackage{dvipsnames,svgnames,x11names}{xcolor}
%
\documentclass[
]{article}
\usepackage{xcolor}
\usepackage{amsmath,amssymb}
\setcounter{secnumdepth}{5}
\usepackage{iftex}
\ifPDFTeX
  \usepackage[T1]{fontenc}
  \usepackage[utf8]{inputenc}
  \usepackage{textcomp} % provide euro and other symbols
\else % if luatex or xetex
  \usepackage{unicode-math} % this also loads fontspec
  \defaultfontfeatures{Scale=MatchLowercase}
  \defaultfontfeatures[\rmfamily]{Ligatures=TeX,Scale=1}
\fi
\usepackage{lmodern}
\ifPDFTeX\else
  % xetex/luatex font selection
\fi
% Use upquote if available, for straight quotes in verbatim environments
\IfFileExists{upquote.sty}{\usepackage{upquote}}{}
\IfFileExists{microtype.sty}{% use microtype if available
  \usepackage[]{microtype}
  \UseMicrotypeSet[protrusion]{basicmath} % disable protrusion for tt fonts
}{}
\makeatletter
\@ifundefined{KOMAClassName}{% if non-KOMA class
  \IfFileExists{parskip.sty}{%
    \usepackage{parskip}
  }{% else
    \setlength{\parindent}{0pt}
    \setlength{\parskip}{6pt plus 2pt minus 1pt}}
}{% if KOMA class
  \KOMAoptions{parskip=half}}
\makeatother
% Make \paragraph and \subparagraph free-standing
\makeatletter
\ifx\paragraph\undefined\else
  \let\oldparagraph\paragraph
  \renewcommand{\paragraph}{
    \@ifstar
      \xxxParagraphStar
      \xxxParagraphNoStar
  }
  \newcommand{\xxxParagraphStar}[1]{\oldparagraph*{#1}\mbox{}}
  \newcommand{\xxxParagraphNoStar}[1]{\oldparagraph{#1}\mbox{}}
\fi
\ifx\subparagraph\undefined\else
  \let\oldsubparagraph\subparagraph
  \renewcommand{\subparagraph}{
    \@ifstar
      \xxxSubParagraphStar
      \xxxSubParagraphNoStar
  }
  \newcommand{\xxxSubParagraphStar}[1]{\oldsubparagraph*{#1}\mbox{}}
  \newcommand{\xxxSubParagraphNoStar}[1]{\oldsubparagraph{#1}\mbox{}}
\fi
\makeatother


\usepackage{longtable,booktabs,array}
\newcounter{none} % for unnumbered tables
\usepackage{calc} % for calculating minipage widths
% Correct order of tables after \paragraph or \subparagraph
\usepackage{etoolbox}
\makeatletter
\patchcmd\longtable{\par}{\if@noskipsec\mbox{}\fi\par}{}{}
\makeatother
% Allow footnotes in longtable head/foot
\IfFileExists{footnotehyper.sty}{\usepackage{footnotehyper}}{\usepackage{footnote}}
\makesavenoteenv{longtable}
\usepackage{graphicx}
\makeatletter
\newsavebox\pandoc@box
\newcommand*\pandocbounded[1]{% scales image to fit in text height/width
  \sbox\pandoc@box{#1}%
  \Gscale@div\@tempa{\textheight}{\dimexpr\ht\pandoc@box+\dp\pandoc@box\relax}%
  \Gscale@div\@tempb{\linewidth}{\wd\pandoc@box}%
  \ifdim\@tempb\p@<\@tempa\p@\let\@tempa\@tempb\fi% select the smaller of both
  \ifdim\@tempa\p@<\p@\scalebox{\@tempa}{\usebox\pandoc@box}%
  \else\usebox{\pandoc@box}%
  \fi%
}
% Set default figure placement to htbp
\def\fps@figure{htbp}
\makeatother


% definitions for citeproc citations
\NewDocumentCommand\citeproctext{}{}
\NewDocumentCommand\citeproc{mm}{%
  \begingroup\def\citeproctext{#2}\cite{#1}\endgroup}
\makeatletter
 % allow citations to break across lines
 \let\@cite@ofmt\@firstofone
 % avoid brackets around text for \cite:
 \def\@biblabel#1{}
 \def\@cite#1#2{{#1\if@tempswa , #2\fi}}
\makeatother
\newlength{\cslhangindent}
\setlength{\cslhangindent}{1.5em}
\newlength{\csllabelwidth}
\setlength{\csllabelwidth}{3em}
\newenvironment{CSLReferences}[2] % #1 hanging-indent, #2 entry-spacing
 {\begin{list}{}{%
  \setlength{\itemindent}{0pt}
  \setlength{\leftmargin}{0pt}
  \setlength{\parsep}{0pt}
  % turn on hanging indent if param 1 is 1
  \ifodd #1
   \setlength{\leftmargin}{\cslhangindent}
   \setlength{\itemindent}{-1\cslhangindent}
  \fi
  % set entry spacing
  \setlength{\itemsep}{#2\baselineskip}}}
 {\end{list}}
\usepackage{calc}
\newcommand{\CSLBlock}[1]{\hfill\break\parbox[t]{\linewidth}{\strut\ignorespaces#1\strut}}
\newcommand{\CSLLeftMargin}[1]{\parbox[t]{\csllabelwidth}{\strut#1\strut}}
\newcommand{\CSLRightInline}[1]{\parbox[t]{\linewidth - \csllabelwidth}{\strut#1\strut}}
\newcommand{\CSLIndent}[1]{\hspace{\cslhangindent}#1}



\setlength{\emergencystretch}{3em} % prevent overfull lines

\providecommand{\tightlist}{%
  \setlength{\itemsep}{0pt}\setlength{\parskip}{0pt}}



 


\makeatletter
\@ifpackageloaded{caption}{}{\usepackage{caption}}
\AtBeginDocument{%
\ifdefined\contentsname
  \renewcommand*\contentsname{Table of contents}
\else
  \newcommand\contentsname{Table of contents}
\fi
\ifdefined\listfigurename
  \renewcommand*\listfigurename{List of Figures}
\else
  \newcommand\listfigurename{List of Figures}
\fi
\ifdefined\listtablename
  \renewcommand*\listtablename{List of Tables}
\else
  \newcommand\listtablename{List of Tables}
\fi
\ifdefined\figurename
  \renewcommand*\figurename{Figure}
\else
  \newcommand\figurename{Figure}
\fi
\ifdefined\tablename
  \renewcommand*\tablename{Table}
\else
  \newcommand\tablename{Table}
\fi
}
\@ifpackageloaded{float}{}{\usepackage{float}}
\floatstyle{ruled}
\@ifundefined{c@chapter}{\newfloat{codelisting}{h}{lop}}{\newfloat{codelisting}{h}{lop}[chapter]}
\floatname{codelisting}{Listing}
\newcommand*\listoflistings{\listof{codelisting}{List of Listings}}
\makeatother
\makeatletter
\makeatother
\makeatletter
\@ifpackageloaded{caption}{}{\usepackage{caption}}
\@ifpackageloaded{subcaption}{}{\usepackage{subcaption}}
\makeatother
\usepackage{bookmark}
\IfFileExists{xurl.sty}{\usepackage{xurl}}{} % add URL line breaks if available
\urlstyle{same}
\hypersetup{
  pdftitle={Model structure conditions ecological inference in food web reconstruction},
  pdfauthor={Tanya Strydom; Baran Karapunar; Andrew P. Beckerman; Alexander Dunhill},
  pdfkeywords={Ecological networks, Biotic interactions, Community
assembly, Environmental gradients, Interaction turnover, Trophic
organisation, Ecosystem resilience, Macroecology},
  colorlinks=true,
  linkcolor={blue},
  filecolor={Maroon},
  citecolor={Blue},
  urlcolor={Blue},
  pdfcreator={LaTeX via pandoc}}



\title{Model structure conditions ecological inference in food web
reconstruction}
\author{Tanya Strydom %
%
\textsuperscript{%
%
1%
}%
; Baran Karapunar %
%
\textsuperscript{%
%
2%
}%
; Andrew P. Beckerman %
%
\textsuperscript{%
%
1%
}%
; Alexander Dunhill %
%
\textsuperscript{%
%
2%
}%
}
\date{2026-02-12}

\usepackage{setspace}
\usepackage[left]{lineno}
\usepackage[letterpaper]{geometry}

\usepackage[nolists,noheads,markers]{endfloat}
\geometry{margin=2.5cm}

\begin{document}

\thispagestyle{empty}
{\bfseries\sffamily\Large Model structure conditions ecological
inference in food web reconstruction}
\vfil
Tanya Strydom %
%
\textsuperscript{%
%
1%
}%
; Baran Karapunar %
%
\textsuperscript{%
%
2%
}%
; Andrew P. Beckerman %
%
\textsuperscript{%
%
1%
}%
; Alexander Dunhill %
%
\textsuperscript{%
%
2%
}%

\vfil
{\small
\textbf{Abstract:} Aim

Ecological networks are widely used to compare community structure,
stability, and responses to disturbance across environmental gradients.
However, many networks---particularly those assembled from incomplete
interaction data---require model-based reconstruction. We test how
alternative reconstruction frameworks condition ecological inference by
quantifying their effects on network structure and disturbance dynamics.

Location

Cleveland Basin, United Kingdom.

Time period

Early Jurassic (upper Pliensbachian--upper Toarcian, \textasciitilde183
Ma).

Major taxa studied

Marine invertebrate communities.

Methods

We reconstructed four successive communities from an identical species
pool using six contrasting food-web models spanning feasible
(trait-based), realised (allometric and energetic), and structural
(topological) network representations. For each community and model, 100
replicate networks were generated. We quantified macro-, meso-, and
micro-scale network properties and assessed differences among models
using multivariate analyses. Pairwise interaction turnover was measured
using link-based beta diversity. We then simulated species loss under
multiple disturbance scenarios, allowing secondary extinctions, and
compared predicted community states using mean absolute differences and
rank concordance metrics.

Results

Reconstruction framework strongly influenced inferred network topology
(MANOVA, p \textless{} 0.001), generating distinct structural signatures
independent of species composition. Models that were similar in global
metrics often diverged in species-level interactions, with high
β-turnover among inferred link sets. During disturbance simulations,
species-level vulnerability rankings were broadly consistent across
models, but interaction-level outcomes and cascade dynamics varied
substantially. Concordance in extinction-scenario rankings was scale
dependent, with higher agreement at the species level than at the
interaction level.

Main conclusions

Network reconstruction functions as a structural prior that conditions
ecological inference. While some aggregate patterns are robust across
modelling frameworks, detailed interaction-level dynamics are highly
model contingent. Comparative network studies across spatial or
environmental gradients should therefore align reconstruction framework
with inferential goals and explicitly evaluate sensitivity to modelling
assumptions.
\vfil
\textbf{Keywords:} %
Ecological networks, Biotic interactions, Community
assembly, Environmental gradients, Interaction turnover, Trophic
organisation, Ecosystem resilience, %
Macroecology%
}
\clearpage
\setcounter{page}{1}
\doublespacing
\linenumbers


\section{Introduction}\label{introduction}

Ecological networks provide a powerful framework for understanding how
communities are structured across space and time. By representing
species and their interactions explicitly, food webs allow ecologists to
quantify complexity, trophic organization, vulnerability, and the
propagation of disturbance through ecosystems (Delmas et al., 2018).
Network approaches have therefore become central to comparative ecology,
from evaluating latitudinal gradients in interaction structure to
assessing how communities reorganize following environmental change
(Gravel et al., 2019; Hao et al., 2025; Poisot et al., 2015; Tylianakis
\& Morris, 2017). However, ecological networks are rarely fully observed
(even in modern systems), and interaction data are incomplete, biased,
and scale dependent (Catchen et al., 2023; Poisot et al., 2021; Sandra
et al., 2025). In most contexts (including historical, biogeographic,
and deep-time systems) interactions must be inferred indirectly from
traits, co-occurrence, phylogeny, or ecological theory (Morales-Castilla
et al., 2015; Strydom et al., 2021). As a result, network construction
is not simply a descriptive exercise but an inferential one where models
are used to predict links that are plausible, probable, or theoretically
consistent with ecological constraints (Strydom et al., 2026). Despite
rapid methodological development in interaction inference, few studies
have systematically compared alternative reconstruction frameworks
within the same empirical system to evaluate how model choice propagates
into ecological inference.

This issue is particularly important for comparative studies. While
inference from ecological networks regarding structure and complexity is
a mathematical task and is therefore relatively independent of
biological assumptions (Delmas et al., 2019), their implications for
stability and dynamics depend critically on assumptions about the
distribution and strength of interaction (Allesina \& Tang, 2012; Poisot
et al., 2015). Network properties such as connectance, trophic
organization, motif frequency, and robustness are often compared across
communities to infer ecological differences attributable to
environmental gradients, disturbance regimes, or evolutionary history
(Dunhill et al., 2024; Michalska-Smith \& Allesina, 2019; Poisot \&
Gravel, 2014; Roopnarine, 2006). However, if network structure depends
strongly on the reconstruction model employed, then methodological
variation may be conflated with biological signal. Understanding which
ecological conclusions are robust to reconstruction assumptions (and
which are model-dependent) is therefore essential for reliable
cross-system inference. Recent work in network ecology has clarified
that reconstruction approaches differ fundamentally in the type of
network that they represent (Gauzens et al., 2025; Strydom et al.,
2026). Broadly, these include: feasible networks, which map the set of
interactions that are biologically possible given trait or phylogenetic
compatibility; realised networks, which incorporate energetic or
foraging constraints to approximate the subset of interactions likely to
occur; and structural networks, which reproduce general topological
properties without assigning biologically explicit species identities
(Allesina et al., 2008). Each representation encodes distinct ecological
assumptions about how interactions arise and persist. Yet these classes
are rarely evaluated comparatively within the same empirical system.

Although modern ecological networks often incorporate direct
observations, analyses across historical or biogeographic gradients rely
on inferred interactions. In these cases, reconstruction becomes
structural hypothesis testing rather than data recovery. Yet most
studies adopt a single reconstruction framework without assessing how
alternative models might alter inferred ecological patterns, leaving it
unclear whether signals such as extinction cascades or stability metrics
reflect ecological reality or modelling artefacts. Deep-time ecosystems
provide a stringent test of this issue, because interactions are not
observed directly (Dunhill et al., 2024; Dunne et al., 2008; Dunne et
al., 2014; Roopnarine, 2006), reconstruction assumptions must be
explicit, allowing model effects on ecological inference to be isolated.

Here we assess how alternative network reconstruction frameworks
influence inferred food web structure and extinction dynamics through a
re-evaluation of primary and secondary, trait-based extinction dynamics
during the early Toarcian extinction event, a volcanic-driven
hyperthermal and marine crisis in the Early Jurassic (\textasciitilde183
Ma) (Dunhill et al., 2024). We reconstruct four successive communities
from an identical taxon pool using six contrasting models spanning
feasible, realised, and structural network representations. For each
community, we compare macro-, meso-, and micro-scale network properties,
quantify turnover in inferred interactions, and evaluate extinction
dynamics under replicated disturbance simulations. In holding species
composition constant while varying reconstruction framework, we isolate
the contribution of model structure to ecological inference. This design
allows us to distinguish signals that are consistent across models
(indicating robust ecological patterns) from those that vary strongly
with reconstruction assumptions. In doing so, we provide a general
framework for evaluating uncertainty in reconstructed ecological
networks and for improving the reliability of comparative network
analyses across spatial and temporal scales.

\section{Materials and Methods}\label{materials-and-methods}

\subsection{Study system and fossil
data}\label{study-system-and-fossil-data}

We used fossil occurrence data from the Cleveland Basin spanning the
upper Pliensbachian to the upper Toarcian, following Dunhill et al.
(2024). Four paleo-communities were defined: pre-extinction,
post-extinction, early recovery, and late recovery. Each assemblage was
treated as a community of potentially interacting taxa. Modes of life
were assigned following Dunhill et al. (2024) using the Bambach ecospace
framework (Bambach et al., 2007). Traits included motility, tiering,
feeding mode, and size class, with size defined categorically based on
maximum linear dimensions.

\subsection{Network reconstruction
approaches}\label{network-reconstruction-approaches}

\subsubsection{Conceptual classification of network
types}\label{conceptual-classification-of-network-types}

Most paleo-specific approaches currently operate within the feasibility
space (\emph{e.g.,} Shaw et al., 2024; Fricke et al., 2022; Roopnarine,
2006). Although well suited for reconstructing feasible interactions,
these methods represent only a subset of the broader spectrum of network
construction approaches. Here, we present a suite of models
(Table~\ref{tbl-models}) that enable the construction of a wider range
of ecological networks and the exploration of a broader set of
ecological questions, provided that their underlying assumptions are
compatible with the constraints of fossil data. For example, some tools
require quantitative estimates of body size, which must often be
inferred from size classes or functional morphology in the fossil
record. Structural models, such as the niche model, rely only on species
richness and estimates/specification of connectance, but their
species-agnostic nature limits their applicability to trait-based or
diet-specific questions, although they do still accurately recover
network structure (Stouffer et al., 2005) . Mechanistic approaches, by
contrast, depend on accurate assignment of feeding traits or robust
phylogenetic support. Recognizing how these methodological requirements
intersect with the limits of fossil evidence is essential for selecting
an appropriate modelling framework.

\begin{longtable}[]{@{}
  >{\raggedright\arraybackslash}p{(\linewidth - 12\tabcolsep) * \real{0.1429}}
  >{\raggedright\arraybackslash}p{(\linewidth - 12\tabcolsep) * \real{0.1429}}
  >{\raggedright\arraybackslash}p{(\linewidth - 12\tabcolsep) * \real{0.1429}}
  >{\raggedright\arraybackslash}p{(\linewidth - 12\tabcolsep) * \real{0.1429}}
  >{\raggedright\arraybackslash}p{(\linewidth - 12\tabcolsep) * \real{0.1429}}
  >{\raggedright\arraybackslash}p{(\linewidth - 12\tabcolsep) * \real{0.1429}}
  >{\raggedright\arraybackslash}p{(\linewidth - 12\tabcolsep) * \real{0.1429}}@{}}
\caption{Six different models that can be used to construct food webs
for both this specific community but are also broadly suited to paleo
network prediction. These models span all facets of the network
representation space (feasibility, realised, and structural network) and
are suitable for an array of different paleo communities as the data
requirements fall within the limitations set by the fossil
record.}\label{tbl-models}\tabularnewline
\toprule\noalign{}
\begin{minipage}[b]{\linewidth}\raggedright
Model family
\end{minipage} & \begin{minipage}[b]{\linewidth}\raggedright
Assumptions
\end{minipage} & \begin{minipage}[b]{\linewidth}\raggedright
Data needs
\end{minipage} & \begin{minipage}[b]{\linewidth}\raggedright
Limitation
\end{minipage} & \begin{minipage}[b]{\linewidth}\raggedright
Network type
\end{minipage} & \begin{minipage}[b]{\linewidth}\raggedright
Key reference
\end{minipage} & \begin{minipage}[b]{\linewidth}\raggedright
Usage examples
\end{minipage} \\
\midrule\noalign{}
\endfirsthead
\toprule\noalign{}
\begin{minipage}[b]{\linewidth}\raggedright
Model family
\end{minipage} & \begin{minipage}[b]{\linewidth}\raggedright
Assumptions
\end{minipage} & \begin{minipage}[b]{\linewidth}\raggedright
Data needs
\end{minipage} & \begin{minipage}[b]{\linewidth}\raggedright
Limitation
\end{minipage} & \begin{minipage}[b]{\linewidth}\raggedright
Network type
\end{minipage} & \begin{minipage}[b]{\linewidth}\raggedright
Key reference
\end{minipage} & \begin{minipage}[b]{\linewidth}\raggedright
Usage examples
\end{minipage} \\
\midrule\noalign{}
\endhead
\bottomrule\noalign{}
\endlastfoot
Random & Links assigned randomly & Species richness, number of links &
Parameter assumptions, species agnostic & Structural & Erdős \& Rényi
(1959) & Null-model comparisons; testing whether observed network
structure (connectance, motifs) deviates from random expectations \\
Niche & Species ordered along a `niche axis'; interactions
interval-constrained & Species richness, connectance & Parameter
assumptions, species agnostic & Structural & Williams \& Martinez (2008)
& Evaluating trophic hierarchy and motif structure; baseline structural
predictions \\
Allometric diet breadth model (ADBM) & Energy-maximizing predator diets
& Body mass, abundance/biomass & Assumes optimal foraging; does not
account for forbidden links & Realised & Petchey et al. (2008) &
Predicting realized predator diets; exploring secondary extinctions \\
Allometric trophic network (ATN) & Links constrained by body-size ratios
and functional response & Body mass, number of basal species & Assumes
only mechanical/energetic constraints & Realised & Brose et al. (2006);
Gauzens et al. (2023) & Simulating species loss; evaluating network
collapse dynamics \\
Paleo food web inference model (PFIM) & Interactions inferred using
trait-based mechanistic rules & Feeding traits & Assumes feeding
mechanisms; trait resolution required & Feasibility & Shaw et al. (2024)
& Mapping feasible trophic interactions; assessing secondary
extinctions \\
Body-size ratio model & Probabilistic assignment of links based on
predator--prey size ratios & Body mass & Does not account for forbidden
links & Realised & Rohr et al. (2010) & Estimating likely interactions;
simulating cascading effects. \\
\end{longtable}

The three body mass-based models (ADBM, ATN, Body-size ratio) differ
primarily in their underlying ecological assumptions. Although all three
models use body mass to infer food web structure, they differ in their
ecological assumptions. The ADBM is based on energy maximization under
optimal foraging theory, the ATN constrains interactions via
mechanically optimal consumer--resource size ratios, and the Body-size
ratio model defines links probabilistically within a fixed allometric
niche. Together, these approaches span bioenergetic, mechanical, and
statistical interpretations of size-structured interactions.

\subsubsection{Network generation and
replication}\label{network-generation-and-replication}

We evaluated six models spanning this space Table~\ref{tbl-models}:
random and niche models (structural network); allometric diet breadth
(ADBM), allometric trophic network (ATN), and Body-size ratio models
(realised network); and a paleo food web inference model (PFIM;
feasibility web). Expanded descriptions of model assumptions,
parameterisation, and link-generation rules are provided in
Supplementary Material S1. For each community, 100 networks were
generated per model (n = 2400) to capture stochastic variation in link
assignment. Where models required species body mass or trait values,
these were sampled within biologically reasonable ranges to preserve
relative differences among species. We adopted uniform sampling by
default, as alternative distributions (lognormal, truncated lognormal)
have negligible impact on topology (Supplementary Material S2; Figure
S1). Structural models were parameterized using connectance values drawn
from an empirically realistic range (0.07--0.34), with species richness
held constant. Identical parameter draws were applied across comparable
models within each replicate to ensure comparability. For the Body-size
ratio model, we followed the approach of Yeakel et al. (2014) and
excluded latent trait terms as opposed fitting the full model, which
introduces additional inference and assumptions.

\subsection{Network metrics and structural
analyses}\label{network-metrics-and-structural-analyses}

We quantified network structure using a suite of macro-, meso-, and
micro-scale metrics Table~\ref{tbl-properties}, capturing global
properties, motif structure, and species-level variability. Differences
among reconstruction approaches were assessed using a multivariate
analysis of variance (MANOVA), with model identity as a fixed factor and
the full set of network metrics as response variables. Pairwise
interaction turnover was quantified using link‑based beta diversity,
which measures dissimilarity in the identity of trophic links between
networks, capturing differences due to species turnover or changes in
interactions among shared species (Poisot et al., 2012).

\begin{longtable}[]{@{}
  >{\raggedright\arraybackslash}p{(\linewidth - 6\tabcolsep) * \real{0.2466}}
  >{\raggedright\arraybackslash}p{(\linewidth - 6\tabcolsep) * \real{0.2603}}
  >{\raggedright\arraybackslash}p{(\linewidth - 6\tabcolsep) * \real{0.2466}}
  >{\raggedright\arraybackslash}p{(\linewidth - 6\tabcolsep) * \real{0.2466}}@{}}
\caption{Network properties used for
analysis.}\label{tbl-properties}\tabularnewline
\toprule\noalign{}
\begin{minipage}[b]{\linewidth}\raggedright
Metric
\end{minipage} & \begin{minipage}[b]{\linewidth}\raggedright
Definition
\end{minipage} & \begin{minipage}[b]{\linewidth}\raggedright
Scale
\end{minipage} & \begin{minipage}[b]{\linewidth}\raggedright
Reference (for maths), can make footnotes probs
\end{minipage} \\
\midrule\noalign{}
\endfirsthead
\toprule\noalign{}
\begin{minipage}[b]{\linewidth}\raggedright
Metric
\end{minipage} & \begin{minipage}[b]{\linewidth}\raggedright
Definition
\end{minipage} & \begin{minipage}[b]{\linewidth}\raggedright
Scale
\end{minipage} & \begin{minipage}[b]{\linewidth}\raggedright
Reference (for maths), can make footnotes probs
\end{minipage} \\
\midrule\noalign{}
\endhead
\bottomrule\noalign{}
\endlastfoot
Richness & Number of nodes in the network & Macro & \\
Links & Normalized standard deviation of links (number of consumers plus
resources per taxon) & Micro & \\
Connectance & \(L/S^2\), where \(S\) is the number of species and \(L\)
the number of links & Macro & \\
Max trophic level & Prey-weighted trophic level averaged across taxa &
Macro & Williams \& Martinez (2004) \\
S1 & Number of linear chains, normalised & Meso & Milo et al. (2002);
Stouffer et al. (2007) \\
S2 & Number of omnivory motifs, normalised & Meso & Milo et al. (2002);
Stouffer et al. (2007) \\
S4 & Number of apparent competition motifs, normalised & Meso & Milo et
al. (2002); Stouffer et al. (2007) \\
S5 & Number of direct competition motifs, normalised & Meso & Milo et
al. (2002); Stouffer et al. (2007) \\
Generality & Normalized standard deviation of generality of a species,
standardised by \(L/S\) & Micro & Williams \& Martinez (2000) \\
Vulnerability & Normalized standard deviation of vulnerability of a
species, standardised by \(L/S\) & Micro & Williams \& Martinez
(2000) \\
\end{longtable}

\subsection{Extinction simulations and model
evaluation}\label{extinction-simulations-and-model-evaluation}

Following Dunhill et al. (2024) and using the pre-extinction and
post-extinction networks, we simulated species loss under multiple
extinction scenarios. We simulated species loss under multiple
scenarios, including trait-based, network-position-based, and random
removals, allowing cascading extinctions to propagate. Simulated
outcomes were compared to observed or expected community states using
mean absolute differences (MAD) and modified true skill statistics (TSS)
at node and link levels. Kendall's rank correlation coefficient was used
to evaluate concordance in scenario rankings across reconstruction
models, providing a measure of robustness in inferred community
responses.

\section{Results}\label{results}

Across six network reconstruction approaches, inferred food web
structure, species interactions, and extinction dynamics differed
consistently. Multivariate analyses showed pronounced separation among
models in network metric space. Reconstruction approach explained most
of the variance in structural properties, leaving a distinct signature
independent of community composition. Notably, agreement among models
depended on scale - approaches that were statistically similar in
multivariate structural space often diverged in inferred interactions or
extinction dynamics. This demonstrates that structural similarity does
not guarantee concordance in species-level diets or trophic roles. Model
choice also substantially influenced inferred extinction dynamics.
Temporal trajectories of network collapse, interaction loss, and motif
reorganization differed among approaches. Although species-level
extinction rankings were often broadly consistent, link-level outcomes
and extinction inferences were highly sensitive to reconstruction
assumptions. Together, these results show that ecological inferences
drawn from paleo networks depend critically on the reconstruction
framework employed. Importantly, agreement among models was not
consistent across analytical scales - models that were statistically
indistinguishable in multivariate structural space often diverged in
inferred interactions or extinction dynamics. Together these results
show that reconstruction approaches that appear similar when evaluated
using global network metrics can yield fundamentally different
ecological narratives when interrogated at the level of interactions and
extinction dynamics.

\subsection{Network structure differs among reconstruction
approaches}\label{network-structure-differs-among-reconstruction-approaches}

Across six reconstruction approaches, network structure differed
significantly (MANOVA, Pillai's trace = 3.84, approximate
\(F_{40,11955}\) = 987.35, p \textless{} 0.001), indicating that model
choice systematically alters inferred food web topology. Canonical
discriminant analysis identified two dominant axes of variation,
explaining 86\% of between-model variance. LD1 correlated with
vulnerability, direct competition motifs, and connectance. LD2
correlated with maximum trophic level and apparent competition motifs,
reflecting vertical trophic structure (Figure 1; Table S1, Figure S1).
All higher-order canonical variates each explained less than 9\% of the
remaining variance.

\begin{figure}

\centering{

\pandocbounded{\includegraphics[keepaspectratio]{figures/MANOVA_lda.png}}

}

\caption{\label{fig-structure}Linear discriminant analysis (LDA) of
ecological network metrics for six model types. Each point represents a
replicate, and ellipses indicate 95\% confidence regions for each model.
The second column represents the correlation of the various network
metrics with the respective LDA axes.}

\end{figure}%

\subsubsection{Inferred pairwise interactions vary widely among
models}\label{inferred-pairwise-interactions-vary-widely-among-models}

Despite some models showing similar global metrics, specific pairwise
interactions often differed. Pairwise β-turnover revealed that certain
model pairs shared very few links Figure~\ref{fig-beta_div}. Size-based
models (ADBM, ATN) were broadly similar due to shared reliance on
body-size constraints, whereas the Body-size ratio model exhibited
consistently higher differences to other models. PFIM showed
intermediate overlap with theoretical models. These results demonstrate
that agreement in global network structure does not guarantee
concordance in species-level interactions.

\begin{figure}

\centering{

\pandocbounded{\includegraphics[keepaspectratio]{figures/beta_div.png}}

}

\caption{\label{fig-beta_div}Pairwise β-turnover in species interactions
among four ecological network models (ADBM, ATN, Body-size ratio, and
pfim). Each cell represents the mean turnover value between a pair of
models, with warmer colours indicating greater dissimilarity in inferred
interactions. The diagonal is omitted. High turnover values (yellow)
indicate strong disagreement in network structure between models,
whereas lower values (blue--purple) indicate greater similarity.}

\end{figure}%

\subsection{Model choice influences inferred extinction
dynamics}\label{model-choice-influences-inferred-extinction-dynamics}

To quantify how network structure changed over time during extinction
simulations and whether these dynamics differed among reconstruction
models, we fit generalized additive models (GAMs) to time series of
network-level metrics. GAMs capture non-linear temporal trajectories,
allowing formal tests of whether the shape of these trajectories differs
among models. These model-specific temporal trajectories are shown in
Figure~\ref{fig-gam}. For all metrics examined, the inclusion of
model-specific smooth terms significantly improved model fit (ANOVA
model comparison: \(p\) \textless{} 0.001 for all metrics).
Model-specific smooths differed not only in magnitude but also in the
timing and abruptness of change, indicating distinct modes of collapse
across reconstruction approaches (Tables S3--S4). Deterministic,
data-driven approaches (PFIM) and allometric models (ADBM, ATN)
exhibited highly non-linear trajectories, showing structural shifts in
connectivity and motif frequency. In contrast, the Niche model produced
the most consistent and gradual trajectories, effectively smoothing the
perceived magnitude of structural change during community collapse.
These results demonstrate that inferred pathways of collapse, trophic
bottlenecks, and secondary extinctions are highly sensitive to model
choice. Corresponding raw temporal trajectories are shown in Figure S2.

\begin{figure}

\centering{

\pandocbounded{\includegraphics[keepaspectratio]{figures/GAM_predictions.png}}

}

\caption{\label{fig-gam}GAM-predicted trajectories of network structure
during extinction simulations reveal pronounced differences in the
timing and magnitude of change across reconstruction models. Lines show
model-specific smooths and shaded areas indicate 95\% confidence
intervals. Deterministic approaches produce smoother, more consistent
dynamics, whereas stochastic models exhibit greater variability,
underscoring the sensitivity of inferred collapse pathways to
reconstruction assumptions.}

\end{figure}%

To evaluate how model choice affects inferred extinction dynamics, we
compared simulated post-extinction networks to observed networks using
mean absolute differences (MAD) for network-level metrics and total
sum-of-squares (TSS) for node- and link-level outcomes
Figure~\ref{fig-mad}. Across models, MAD-based rankings were generally
positively correlated (Kendall's \(\tau\) ≈ 0.13 across structural
metrics), indicating broad agreement on the relative importance of
extinction drivers despite substantial differences in reconstructed
network structure. However, agreement within the allometric models
differed from patterns observed for reconstructed network structure.
Whereas earlier multivariate analyses showed strongest structural
similarity between the ADBM and Body-size ratio models,
extinction-driven network responses aligned most closely between the
ADBM and ATN models (mean \(\tau\) ≈ 0.67 across structural metrics),
with little correspondence between ADBM and Body-size ratio model
outcomes (mean \(\tau\) ≈ 0.05). This reversal relative to structural
similarity analyses demonstrates that model concordance is context
dependent, with emergent topology and extinction dynamics emphasizing
different aspects of model assumptions. Node-level TSS rankings were
similarly consistent across models (\(\tau\) = 0.26--0.90), reflecting
broadly comparable species removal sequences. In contrast, link-level
outcomes were far more variable (\(\tau\) = −0.48--0.29), highlighting
that inferences about which interactions are lost, retained, or
re-established during collapse and recovery are highly model contingent.
Together, these results suggest that while alternative models converge
on similar species-level extinction patterns, the inferred pathways of
interaction loss and cascading dynamics depend strongly on both
reconstruction approach.

\begin{figure}

\centering{

\pandocbounded{\includegraphics[keepaspectratio]{figures/kendal_tau.png}}

}

\caption{\label{fig-mad}Heatmaps showing pairwise Kendall rank
correlation coefficients (\(\tau\)) between models for each network
metric. Each panel corresponds to a different metric and displays the
degree of agreement in extinction-scenario rankings across models based
on mean absolute differences (MAD) between observed and predicted
network values. Positive \(\tau\) values (blue) indicate concordant
rankings between models, whereas negative \(\tau\) values (red) indicate
opposing rankings. Warmer colours approaching zero represent little or
no agreement. Panels illustrate how consistently different modelling
approaches evaluate the relative realism of extinction scenarios across
multiple network properties.}

\end{figure}%

\section{Discussion}\label{discussion}

\subsection{Model choice as a component of ecological
inference}\label{model-choice-as-a-component-of-ecological-inference}

Reconstructing ecological networks from incomplete interaction data
(whether in contemporary, historical, or deep-time systems) is
fundamentally an exercise in inference under uncertainty. Even in modern
ecosystems, interaction networks are rarely fully observed (Poisot et
al., 2021), and link prediction often relies on traits, phylogeny,
co-occurrence, or mechanistic assumptions (Delmas et al., 2019;
Morales-Castilla et al., 2015; Strydom et al., 2026). Theoretical work
has long demonstrated that network structure strongly conditions
ecological dynamics, including robustness to species loss and the
propagation of disturbance (Allesina \& Tang, 2012; Dunne et al., 2002;
Solé \& Montoya, 2001). Our results extend this insight by showing that
the reconstruction framework itself functions as a structural prior - in
shaping interaction topology, it directly influences inferred food web
organisation and community responses to disturbance.

Differences among models arise not from the species pool alone, but from
assumptions embedded in each model family (Pichler \& Hartig, 2023;
Strydom et al., 2021; Strydom et al., 2026). These include how trophic
links are defined (trait compatibility versus energetic optimisation),
how interaction probabilities are parameterised, and whether topology is
constrained by macroecological regularities (\emph{e.g.,} niche
structure) or mechanistic rules (\emph{e.g.,} body-size scaling).
Consequently, network reconstruction is not a neutral technical step; it
encodes ecological hypotheses that shape both emergent structure and
dynamical predictions. This sensitivity parallels challenges in
contemporary network ecology, where model and metric selection influence
interpretations of connectance, modularity, motif frequencies, and
stability (Michalska-Smith \& Allesina, 2019; Poisot \& Gravel, 2014).

Multivariate analyses revealed that reconstruction approaches differ
along a small number of dominant axes corresponding to horizontal
interaction density and vertical trophic organisation. These axes
capture systematic, model-specific signatures independent of species
composition, indicating that reconstruction framework acts as a major
determinant of inferred ecological structure. Although some models
converged on global metrics (\emph{e.g.,} ADBM and ATN models), pairwise
β-turnover revealed disagreements in inferred species-level
interactions. Structural similarity therefore does not guarantee
concordance in trophic roles. Importantly, extinction scenario inference
was scale dependent. Species-level extinction rankings were relatively
consistent across models, whereas interaction-level outcomes were highly
sensitive to reconstruction assumptions. This asymmetry reflects the
dependence of cascade dynamics on link configuration and interaction
distribution (Allesina \& Tang, 2012; Curtsdotter et al., 2011; Dunne et
al., 2002). Thus, while certain aggregate patterns may be robust to
modelling choices, fine-grained interaction-level inference remains
intrinsically model contingent.

Taken together, these results underscore that network reconstruction is
a hypothesis-generating process where each model encodes a distinct set
of ecological assumptions, and the inferred structure and dynamics
reflect these assumptions. Accordingly, researchers should carefully
align reconstruction approaches with the specific ecological signals of
interest, whether potential interactions, realised diets, or macro scale
structural properties. Disagreement among models does not imply that any
single approach is `wrong', but rather that different models capture
different facets of ecological reality (Stouffer, 2019). Viewed through
the lens of accuracy (here referring to model convergence/robustness)
and precision, our results suggest that some paleoecological inferences
are robust across reconstruction assumptions, while others remain
intrinsically uncertain. Models consistently recover similar high-level
extinction patterns, implying relative accuracy, but disagree on
interaction-level details and temporal dynamics, indicating limited
precision in reconstructing the fine structure of collapse. Recognizing
and explicitly accounting for these differences is essential for
advancing paleoecology beyond descriptive reconstruction toward rigorous
comparative inference.

\subsection{Matching ecological questions to network
representations}\label{matching-ecological-questions-to-network-representations}

A central implication is that network representations are question
specific. Different ecological questions require different classes of
network models, a distinction increasingly recognised in contemporary
ecology (Gauzens et al., 2025; Gravel et al., 2013; Tylianakis \&
Morris, 2017).

\textbf{Feasibility networks:} (trait- or phylogeny-based metaweb
approaches) delineate the set of biologically plausible interactions.
These are well suited for investigating potential dietary breadth,
interaction diversity, or assembly constraints across spatial or
environmental gradients (Gravel et al., 2019). However, because they
maximise compatibility rather than realised foraging dynamics, they may
overestimate interaction density when used to infer cascade processes.

\textbf{Realised networks:} (allometric or energetic models such as ADBM
and ATN) embed foraging and metabolic rules to approximate likely
trophic interactions (Brose et al., 2006; Gauzens et al., 2023; Petchey
et al., 2008). In our analyses, these models produced more nonlinear and
abrupt disturbance trajectories, consistent with energetic bottlenecks
and constraint propagation. They are therefore more appropriate for
questions concerning energy flow, trophic stability, and secondary
extinction dynamics.

\textbf{Structural networks:} (such as the niche or cascade models)
prioritise topological regularities over species identity (Allesina et
al., 2008; Williams \& Martinez, 2008). These approaches are
particularly useful when evaluating macroecological scaling
relationships, connectance patterns, motif distributions, or theoretical
expectations for network structure. However, because species identity is
decoupled from interaction assignment, they are less suitable for
species-specific ecological inference.

Rather than asking which model is `correct', the more productive
question is which representation best aligns with the inferential goal.
Network reconstruction should therefore be treated as part of hypothesis
specification, not merely data preparation.

\subsection{Implications for ecological network
analysis}\label{implications-for-ecological-network-analysis}

Although the present analyses were conducted within a single regional
species pool, the implications extend broadly to comparative ecology and
biogeography. Networks assembled across environmental gradients,
latitudinal bands, disturbance regimes, or temporal intervals often
differ in sampling intensity, trait resolution, and reconstruction
methodology (Delmas et al., 2018; Poisot et al., 2021; Tylianakis \&
Morris, 2017). Without explicitly accounting for reconstruction
framework, methodological variation may be conflated with ecological
signal.

Three general implications follow.

First, reconstruction assumptions should be treated as explicit
components of study design. Because network models encode hypotheses
about how interactions arise, observed differences in connectance,
trophic organisation, or robustness may reflect structural priors rather
than ecological processes.

Second, cross-system comparisons should standardise reconstruction
framework wherever possible. Comparing networks generated using
different model families risks attributing differences in structure or
stability to environmental gradients when they may instead arise from
modelling choices.

Third, ensemble or sensitivity-based approaches provide a pathway to
more robust inference. Evaluating ecological patterns across multiple
plausible reconstructions allows identification of signals that are
consistent across models and those that are assumption dependent. In
this study, species-level vulnerability patterns were comparatively
robust, whereas interaction-level cascades were highly variable. Such
scale-dependent robustness clarifies where ecological inference is
reliable and where it remains uncertain.

These considerations are particularly relevant for global change
research. As ecological communities reorganise under climate change,
habitat loss, and species invasions, reconstructed or partially observed
networks are increasingly used to infer vulnerability, tipping points,
and resilience (Michalska-Smith \& Allesina, 2019; Tylianakis \& Morris,
2017). Recognising reconstruction framework as a structural prior
strengthens interpretation of such comparative analyses.

\subsection{Toward probabilistic and ensemble reconstruction
frameworks}\label{toward-probabilistic-and-ensemble-reconstruction-frameworks}

Advances in modern network ecology offer promising directions for
explicitly incorporating uncertainty into reconstruction. Probabilistic
and Bayesian link-prediction approaches allow interaction probabilities
to be estimated rather than assumed deterministic (Baskerville et al.,
2011; Elmasri et al., 2020; Poisot et al., 2016). Maximum entropy
methods can infer network structure under incomplete information while
constraining macroecological properties (Banville et al., 2023).
Trait-based and joint species distribution approaches integrate
environmental, phylogenetic, and functional information to improve link
inference across gradients (Bartomeus et al., 2016; Ovaskainen et al.,
2017). Adopting such approaches would allow reconstructed networks to be
treated as probabilistic ensembles rather than fixed topologies,
improving both transparency and robustness (Banville et al., 2025;
Perez-Lamarque et al., 2026; Poisot et al., 2016). In this framework,
variation among reconstruction models becomes a quantifiable component
of uncertainty rather than a hidden source of bias.

\section{Conclusions}\label{conclusions}

Ecological network reconstruction is not merely a technical step but a
fundamental component of ecological inference. By comparing six
contrasting reconstruction frameworks applied to an identical species
pool, we demonstrate that model choice strongly shapes inferred food-web
structure, interaction identity, and disturbance dynamics. Broad
species-level patterns may be robust across reconstruction approaches,
but interaction-level outcomes and cascade pathways are highly
contingent on model assumptions. These findings highlight that network
reconstruction is inherently hypothesis-driven. Each model encodes
distinct ecological assumptions that influence both emergent topology
and dynamical predictions. No single representation captures all aspects
of ecological reality. However, aligning reconstruction framework with
inferential goals, standardising methods across comparisons, and
adopting ensemble or sensitivity approaches can distinguish robust
ecological signals from model-dependent artefacts. As ecological network
analyses continue to expand across spatial, temporal, and environmental
gradients, recognising reconstruction framework as a structural prior
will be essential for strengthening the reliability and interpretability
of comparative ecological research.

\section*{References}\label{references}
\addcontentsline{toc}{section}{References}

\protect\phantomsection\label{refs}
\begin{CSLReferences}{1}{0}
\bibitem[\citeproctext]{ref-allesina2008}
Allesina, S., Alonso, D., \& Pascual, M. (2008). A general model for
food web structure. \emph{Science}, \emph{320}(5876), 658--661.
\url{https://doi.org/10.1126/science.1156269}

\bibitem[\citeproctext]{ref-allesina2012}
Allesina, S., \& Tang, S. (2012). Stability criteria for complex
ecosystems. \emph{Nature}, \emph{483}(7388), 205--208.
\url{https://doi.org/10.1038/nature10832}

\bibitem[\citeproctext]{ref-bambach2007}
Bambach, R. K., Bush, A. M., \& Erwin, D. H. (2007). Autecology and the
filling of ecospace: Key metazoan radiations. \emph{Palaeontology},
\emph{50}(1), 1--22.
\url{https://doi.org/10.1111/j.1475-4983.2006.00611.x}

\bibitem[\citeproctext]{ref-banville2023}
Banville, F., Gravel, D., \& Poisot, T. (2023). What constrains food
webs? A maximum entropy framework for predicting their structure with
minimal biases. \emph{PLOS Computational Biology}, \emph{19}(9),
e1011458. \url{https://doi.org/10.1371/journal.pcbi.1011458}

\bibitem[\citeproctext]{ref-banville2025}
Banville, F., Strydom, T., Blyth, P. S. A., Brimacombe, C., Catchen, M.
D., Dansereau, G., Higino, G., Malpas, T., Mayall, H., Norman, K.,
Gravel, D., \& Poisot, T. (2025). Deciphering probabilistic species
interaction networks. \emph{Ecology Letters}, \emph{28}(6), e70161.
\url{https://doi.org/10.1111/ele.70161}

\bibitem[\citeproctext]{ref-bartomeus2016}
Bartomeus, I., Gravel, D., Tylianakis, J. M., Aizen, M. A., Dickie, I.
A., \& Bernard-Verdier, M. (2016). A common framework for identifying
linkage rules across different types of interactions. \emph{Functional
Ecology}, \emph{30}(12), 1894--1903.
\url{https://doi.org/10.1111/1365-2435.12666}

\bibitem[\citeproctext]{ref-baskerville2011}
Baskerville, E. B., Dobson, A. P., Bedford, T., Allesina, S., Anderson,
T. M., \& Pascual, M. (2011). Spatial guilds in the serengeti food web
revealed by a bayesian group model. \emph{PLOS Computational Biology},
\emph{7}(12), e1002321.
\url{https://doi.org/10.1371/journal.pcbi.1002321}

\bibitem[\citeproctext]{ref-brose2006}
Brose, U., Jonsson, T., Berlow, E. L., Warren, P., Banasek-Richter, C.,
Bersier, L.-F., Blanchard, J. L., Brey, T., Carpenter, S. R.,
Blandenier, M.-F. C., Cushing, L., Dawah, H. A., Dell, T., Edwards, F.,
Harper-Smith, S., Jacob, U., Ledger, M. E., Martinez, N. D., Memmott,
J., \ldots{} Cohen, J. E. (2006). Consumer{\textendash}resource
body-size relationships in natural food webs. \emph{Ecology},
\emph{87}(10), 2411--2417.
https://doi.org/\url{https://doi.org/10.1890/0012-9658(2006)87\%5B2411:CBRINF\%5D2.0.CO;2}

\bibitem[\citeproctext]{ref-catchen2023}
Catchen, M. D., Lin, M., Poisot, T., Rolnick, D., \& Gonzalez, A.
(2023). \emph{Improving ecological connectivity assessments with
transfer learning and function approximation}.
\url{https://ecoevorxiv.org/repository/view/5348/}

\bibitem[\citeproctext]{ref-curtsdotter2011}
Curtsdotter, A., Binzer, A., Brose, U., De Castro, F., Ebenman, B.,
Eklöf, A., Riede, J. O., Thierry, A., \& Rall, B. C. (2011). Robustness
to secondary extinctions: Comparing trait-based sequential deletions in
static and dynamic food webs. \emph{Basic and Applied Ecology},
\emph{12}(7), 571--580. \url{https://doi.org/10.1016/j.baae.2011.09.008}

\bibitem[\citeproctext]{ref-delmas2018}
Delmas, E., Besson, M., Brice, M.-H., Burkle, L. A., Dalla Riva, G. V.,
Fortin, M.-J., Gravel, D., Guimarães, P. R., Hembry, D. H., Newman, E.
A., Olesen, J. M., Pires, M. M., Yeakel, J. D., \& Poisot, T. (2018).
Analysing ecological networks of species interactions. \emph{Biological
Reviews}, 112540. \url{https://doi.org/10.1111/brv.12433}

\bibitem[\citeproctext]{ref-delmas2019}
Delmas, E., Besson, M., Brice, M.-H., Burkle, L. A., Riva, G. V. D.,
Fortin, M.-J., Gravel, D., Guimarães, P. R., Hembry, D. H., Newman, E.
A., Olesen, J. M., Pires, M. M., Yeakel, J. D., \& Poisot, T. (2019).
Analysing ecological networks of species interactions. \emph{Biological
Reviews}, \emph{94}(1), 16--36. \url{https://doi.org/10.1111/brv.12433}

\bibitem[\citeproctext]{ref-dunhill2024}
Dunhill, A. M., Zarzyczny, K., Shaw, J. O., Atkinson, J. W., Little, C.
T. S., \& Beckerman, A. P. (2024). Extinction cascades, community
collapse, and recovery across a mesozoic hyperthermal event.
\emph{Nature Communications}, \emph{15}(1), 8599.
\url{https://doi.org/10.1038/s41467-024-53000-2}

\bibitem[\citeproctext]{ref-dunne2014}
Dunne, J. A., Labandeira, C. C., \& Williams, R. J. (2014). Highly
resolved early eocene food webs show development of modern trophic
structure after the end-cretaceous extinction. \emph{Proceedings of the
Royal Society B: Biological Sciences}, \emph{281}(1782), 20133280.
\url{https://doi.org/10.1098/rspb.2013.3280}

\bibitem[\citeproctext]{ref-dunne2008}
Dunne, J. A., Williams, R. J., Martinez, N. D., Wood, R. A., \& Erwin,
D. H. (2008). Compilation and network analyses of cambrian food webs.
\emph{PLOS Biology}, \emph{6}(4), e102.
\url{https://doi.org/10.1371/journal.pbio.0060102}

\bibitem[\citeproctext]{ref-dunne2002}
Dunne, J., Williams, R. J., \& Martinez, N. D. (2002). Network structure
and biodiversity loss in food webs: Robustness increases with
connectance. \emph{Ecol. Lett.}, \emph{5}(4), 558--567.

\bibitem[\citeproctext]{ref-elmasri2020}
Elmasri, M., Farrell, M. J., Davies, T. J., \& Stephens, D. A. (2020). A
hierarchical bayesian model for predicting ecological interactions using
scaled evolutionary relationships. \emph{The Annals of Applied
Statistics}, \emph{14}(1), 221--240.
\url{https://doi.org/10.1214/19-AOAS1296}

\bibitem[\citeproctext]{ref-erdos1959}
Erdős, P., \& Rényi, A. (1959). On random graphs. i. \emph{Publicationes
Mathematicae Debrecen}, \emph{6}(3-4), 290--297.
\url{https://doi.org/10.5486/pmd.1959.6.3-4.12}

\bibitem[\citeproctext]{ref-fricke2022a}
Fricke, E. C., Hsieh, C., Middleton, O., Gorczynski, D., Cappello, C.
D., Sanisidro, O., Rowan, J., Svenning, J.-C., \& Beaudrot, L. (2022).
Collapse of terrestrial mammal food webs since the late pleistocene.
\emph{Science}, \emph{377}(6609), 1008--1011.
\url{https://doi.org/10.1126/science.abn4012}

\bibitem[\citeproctext]{ref-gauzens2023}
Gauzens, B., Brose, U., Delmas, E., \& Berti, E. (2023). ATNr:
Allometric trophic network models in r. \emph{Methods in Ecology and
Evolution}, \emph{14}(11), 2766--2773.
\url{https://doi.org/10.1111/2041-210X.14212}

\bibitem[\citeproctext]{ref-gauzens2025}
Gauzens, B., Thouvenot, L., Srivastava, D. S., Kratina, P., Romero, G.
Q., Berti, E., O'Gorman, E. J., González, A. L., Dézerald, O.,
Eisenhauer, N., Pires, M., Ryser, R., Farjalla, V. F., Rogy, P., Brose,
U., Petermann, J. S., Geslin, B., \& Hines, J. (2025). Tailoring
interaction network types to answer different ecological questions.
\emph{Nature Reviews Biodiversity}, 1--10.
\url{https://doi.org/10.1038/s44358-025-00056-7}

\bibitem[\citeproctext]{ref-gravel2019}
Gravel, D., Baiser, B., Dunne, J. A., Kopelke, J.-P., Martinez, N. D.,
Nyman, T., Poisot, T., Stouffer, D. B., Tylianakis, J. M., Wood, S. A.,
\& Roslin, T. (2019). Bringing elton and grinnell together: A
quantitative framework to represent the biogeography of ecological
interaction networks. \emph{Ecography}, \emph{42}(3), 401--415.
https://doi.org/\url{https://doi.org/10.1111/ecog.04006}

\bibitem[\citeproctext]{ref-gravel2013}
Gravel, D., Poisot, T., Albouy, C., Velez, L., \& Mouillot, D. (2013).
Inferring food web structure from predator{\textendash}prey body size
relationships. \emph{Methods in Ecology and Evolution}, \emph{4}(11),
1083--1090.
https://doi.org/\url{https://doi.org/10.1111/2041-210X.12103}

\bibitem[\citeproctext]{ref-hao2025}
Hao, X., Holyoak, M., Zhang, Z., \& Yan, C. (2025). Global projection of
terrestrial vertebrate food webs under future climate and land-use
changes. \emph{Global Change Biology}, \emph{31}(2), e70061.
\url{https://doi.org/10.1111/gcb.70061}

\bibitem[\citeproctext]{ref-michalska-smith2019}
Michalska-Smith, M. J., \& Allesina, S. (2019). Telling ecological
networks apart by their structure: A computational challenge. \emph{PLOS
Computational Biology}, \emph{15}(6), e1007076.
\url{https://doi.org/10.1371/journal.pcbi.1007076}

\bibitem[\citeproctext]{ref-milo2002}
Milo, R., Shen-Orr, S., Itzkovitz, S., Kashtan, N., Chklovskii, D., \&
Alon, U. (2002). Network motifs: Simple building blocks of complex
networks. \emph{Science}, \emph{298}(5594), 824--827.
\url{https://doi.org/10.1126/science.298.5594.824}

\bibitem[\citeproctext]{ref-morales-castilla2015}
Morales-Castilla, I., Matias, M. G., Gravel, D., \& Araújo, M. B.
(2015). Inferring biotic interactions from proxies. \emph{Trends in
Ecology \& Evolution}, \emph{30}(6), 347--356.
\url{https://doi.org/10.1016/j.tree.2015.03.014}

\bibitem[\citeproctext]{ref-ovaskainen2017}
Ovaskainen, O., Tikhonov, G., Norberg, A., Blanchet, F. G., Duan, L.,
Dunson, D., Roslin, T., \& Abrego, N. (2017). How to make more out of
community data? A conceptual framework and its implementation as models
and software. \emph{Ecology Letters}, \emph{20}(5), 561--576.
\url{https://doi.org/10.1111/ele.12757}

\bibitem[\citeproctext]{ref-perez-lamarque2026}
Perez-Lamarque, B., Andréoletti, J., Morillon, B., Pion-Piola, O.,
Lambert, A., \& Morlon, H. (2026). Darwin{'}s entangled bank through
deep time: Structural stability of mutualistic networks over large
geographic and temporal scales. \emph{EcoEvoRxiv}.
\url{https://doi.org/10.1101/2025.10.08.681159}

\bibitem[\citeproctext]{ref-petchey2008}
Petchey, O. L., Beckerman, A. P., Riede, J. O., \& Warren, P. H. (2008).
Size, foraging, and food web structure. \emph{Proceedings of the
National Academy of Sciences}, \emph{105}(11), 4191--4196.
\url{https://doi.org/10.1073/pnas.0710672105}

\bibitem[\citeproctext]{ref-pichler2023}
Pichler, M., \& Hartig, F. (2023). Machine learning and deep
learning{\textemdash}a review for ecologists. \emph{Methods in Ecology
and Evolution}, \emph{14}(4), 994--1016.
\url{https://doi.org/10.1111/2041-210X.14061}

\bibitem[\citeproctext]{ref-poisot2021}
Poisot, T., Bergeron, G., Cazelles, K., Dallas, T., Gravel, D.,
MacDonald, A., Mercier, B., Violet, C., \& Vissault, S. (2021). Global
knowledge gaps in species interaction networks data. \emph{Journal of
Biogeography}, jbi.14127. \url{https://doi.org/10.1111/jbi.14127}

\bibitem[\citeproctext]{ref-poisot2012a}
Poisot, T., Canard, E., Mouillot, D., Mouquet, N., \& Gravel, D. (2012).
The dissimilarity of species interaction networks. \emph{Ecology
Letters}, \emph{15}(12), 1353--1361.
\url{https://doi.org/10.1111/ele.12002}

\bibitem[\citeproctext]{ref-poisot2016}
Poisot, T., Cirtwill, A., Cazelles, K., Gravel, D., Fortin, M.-J., \&
Stouffer, D. (2016). The structure of probabilistic networks.
\emph{Methods in Ecology and Evolution}, \emph{7}(3), 303312.
\url{https://doi.org/10}

\bibitem[\citeproctext]{ref-poisot2014}
Poisot, T., \& Gravel, D. (2014). When is an ecological network complex?
Connectance drives degree distribution and emerging network properties.
\emph{PeerJ}, \emph{2}, e251. \url{https://doi.org/10.7717/peerj.251}

\bibitem[\citeproctext]{ref-poisot2015}
Poisot, T., Stouffer, D. B., \& Gravel, D. (2015). Beyond species: Why
ecological interaction networks vary through space and time.
\emph{Oikos}, \emph{124}(3), 243--251.
\url{https://doi.org/10.1111/oik.01719}

\bibitem[\citeproctext]{ref-rohr2010}
Rohr, R., Scherer, H., Kehrli, P., Mazza, C., \& Bersier, L.-F. (2010).
Modeling food webs: Exploring unexplained structure using latent traits.
\emph{The American Naturalist}, \emph{176}(2), 170--177.
\url{https://doi.org/10.1086/653667}

\bibitem[\citeproctext]{ref-roopnarine2006}
Roopnarine, P. D. (2006). Extinction cascades and catastrophe in ancient
food webs. \emph{Paleobiology}, \emph{32}(1), 1--19.
\url{https://www.jstor.org/stable/4096814}

\bibitem[\citeproctext]{ref-sandra2025}
Sandra, H.-P., Traveset, A., Nogales, M., Heleno, R., Llewelyn, J., \&
Strona, G. (2025). Sampling biases across interaction types affect the
robustness of ecological multilayer networks. \emph{Ecological
Informatics}, 103183. \url{https://doi.org/10.1016/j.ecoinf.2025.103183}

\bibitem[\citeproctext]{ref-shaw2024}
Shaw, J. O., Dunhill, A. M., Beckerman, A. P., Dunne, J. A., \& Hull, P.
M. (2024). \emph{A framework for reconstructing ancient food webs using
functional trait data} (p. 2024.01.30.578036). bioRxiv.
\url{https://doi.org/10.1101/2024.01.30.578036}

\bibitem[\citeproctext]{ref-soluxe92001}
Solé, R. V., \& Montoya, M. (2001). Complexity and fragility in
ecological networks. \emph{Proceedings of the Royal Society of London.
Series B: Biological Sciences}, \emph{268}(1480), 2039--2045.
\url{https://doi.org/10.1098/rspb.2001.1767}

\bibitem[\citeproctext]{ref-stouffer2005}
Stouffer, D. B., Camacho, J., Guimerà, R., Ng, C. A., \& Nunes Amaral,
L. A. (2005). Quantitative patterns in the structure of model and
empirical food webs. \emph{Ecology}, \emph{86}(5), 1301--1311.
\url{https://doi.org/10.1890/04-0957}

\bibitem[\citeproctext]{ref-stouffer2019}
Stouffer, D. B. (2019). All ecological models are wrong, but some are
useful. \emph{Journal of Animal Ecology}, \emph{88}(2), 192--195.
https://doi.org/\url{https://doi.org/10.1111/1365-2656.12949}

\bibitem[\citeproctext]{ref-stouffer2007}
Stouffer, D. B., Camacho, J., Jiang, W., \& Nunes Amaral, L. A. (2007).
Evidence for the existence of a robust pattern of prey selection in food
webs. \emph{Proceedings of the Royal Society B: Biological Sciences},
\emph{274}(1621), 1931--1940.
\url{https://doi.org/10.1098/rspb.2007.0571}

\bibitem[\citeproctext]{ref-strydom2021}
Strydom, T., Catchen, M. D., Banville, F., Caron, D., Dansereau, G.,
Desjardins-Proulx, P., Forero-Muñoz, N. R., Higino, G., Mercier, B.,
Gonzalez, A., Gravel, D., Pollock, L., \& Poisot, T. (2021). A roadmap
towards predicting species interaction networks (across space and time).
\emph{Philosophical Transactions of the Royal Society B: Biological
Sciences}, \emph{376}(1837), 20210063.
\url{https://doi.org/10.1098/rstb.2021.0063}

\bibitem[\citeproctext]{ref-strydom2026}
Strydom, T., Dunhill, A. M., Dunne, J. A., Poisot, T., \& Beckerman, A.
P. (2026). Scaling from metawebs to realised webs: A hierarchical
approach to network ecology. \emph{EcoEvoRxiv}.
\url{https://doi.org/10.32942/X2JW8K}

\bibitem[\citeproctext]{ref-tylianakis2017}
Tylianakis, J. M., \& Morris, R. J. (2017). Ecological networks across
environmental gradients. \emph{Annual Review of Ecology, Evolution, and
Systematics}, \emph{48}(1), 25--48.
\url{https://doi.org/10.1146/annurev-ecolsys-110316-022821}

\bibitem[\citeproctext]{ref-williams2004}
Williams, R. J., \& Martinez, N. D. (2004). Stabilization of chaotic and
non-permanent food-web dynamics. \emph{The European Physical Journal B -
Condensed Matter}, \emph{38}(2), 297--303.
\url{https://doi.org/10.1140/epjb/e2004-00122-1}

\bibitem[\citeproctext]{ref-williams2000}
Williams, R. J., \& Martinez, N. D. (2000). Simple rules yield complex
food webs. \emph{Nature}, \emph{404}(6774), 180--183.
\url{https://doi.org/10.1038/35004572}

\bibitem[\citeproctext]{ref-williams2008}
Williams, R. J., \& Martinez, N. D. (2008). Success and its limits among
structural models of complex food webs. \emph{The Journal of Animal
Ecology}, \emph{77}(3), 512--519.
\url{https://doi.org/10.1111/j.1365-2656.2008.01362.x}

\bibitem[\citeproctext]{ref-yeakel2014}
Yeakel, J. D., Pires, M. M., Rudolf, L., Dominy, N. J., Koch, P. L.,
Guimarães, P. R., \& Gross, T. (2014). Collapse of an ecological network
in ancient egypt. \emph{PNAS}, \emph{111}(40), 14472--14477.
\url{https://doi.org/10.1073/pnas.1408471111}

\end{CSLReferences}





\end{document}
