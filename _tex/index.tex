% Options for packages loaded elsewhere
% Options for packages loaded elsewhere
\PassOptionsToPackage{unicode}{hyperref}
\PassOptionsToPackage{hyphens}{url}
\PassOptionsToPackage{dvipsnames,svgnames,x11names}{xcolor}
%
\documentclass[
]{article}
\usepackage{xcolor}
\usepackage{amsmath,amssymb}
\setcounter{secnumdepth}{5}
\usepackage{iftex}
\ifPDFTeX
  \usepackage[T1]{fontenc}
  \usepackage[utf8]{inputenc}
  \usepackage{textcomp} % provide euro and other symbols
\else % if luatex or xetex
  \usepackage{unicode-math} % this also loads fontspec
  \defaultfontfeatures{Scale=MatchLowercase}
  \defaultfontfeatures[\rmfamily]{Ligatures=TeX,Scale=1}
\fi
\usepackage{lmodern}
\ifPDFTeX\else
  % xetex/luatex font selection
\fi
% Use upquote if available, for straight quotes in verbatim environments
\IfFileExists{upquote.sty}{\usepackage{upquote}}{}
\IfFileExists{microtype.sty}{% use microtype if available
  \usepackage[]{microtype}
  \UseMicrotypeSet[protrusion]{basicmath} % disable protrusion for tt fonts
}{}
\makeatletter
\@ifundefined{KOMAClassName}{% if non-KOMA class
  \IfFileExists{parskip.sty}{%
    \usepackage{parskip}
  }{% else
    \setlength{\parindent}{0pt}
    \setlength{\parskip}{6pt plus 2pt minus 1pt}}
}{% if KOMA class
  \KOMAoptions{parskip=half}}
\makeatother
% Make \paragraph and \subparagraph free-standing
\makeatletter
\ifx\paragraph\undefined\else
  \let\oldparagraph\paragraph
  \renewcommand{\paragraph}{
    \@ifstar
      \xxxParagraphStar
      \xxxParagraphNoStar
  }
  \newcommand{\xxxParagraphStar}[1]{\oldparagraph*{#1}\mbox{}}
  \newcommand{\xxxParagraphNoStar}[1]{\oldparagraph{#1}\mbox{}}
\fi
\ifx\subparagraph\undefined\else
  \let\oldsubparagraph\subparagraph
  \renewcommand{\subparagraph}{
    \@ifstar
      \xxxSubParagraphStar
      \xxxSubParagraphNoStar
  }
  \newcommand{\xxxSubParagraphStar}[1]{\oldsubparagraph*{#1}\mbox{}}
  \newcommand{\xxxSubParagraphNoStar}[1]{\oldsubparagraph{#1}\mbox{}}
\fi
\makeatother


\usepackage{longtable,booktabs,array}
\newcounter{none} % for unnumbered tables
\usepackage{calc} % for calculating minipage widths
% Correct order of tables after \paragraph or \subparagraph
\usepackage{etoolbox}
\makeatletter
\patchcmd\longtable{\par}{\if@noskipsec\mbox{}\fi\par}{}{}
\makeatother
% Allow footnotes in longtable head/foot
\IfFileExists{footnotehyper.sty}{\usepackage{footnotehyper}}{\usepackage{footnote}}
\makesavenoteenv{longtable}
\usepackage{graphicx}
\makeatletter
\newsavebox\pandoc@box
\newcommand*\pandocbounded[1]{% scales image to fit in text height/width
  \sbox\pandoc@box{#1}%
  \Gscale@div\@tempa{\textheight}{\dimexpr\ht\pandoc@box+\dp\pandoc@box\relax}%
  \Gscale@div\@tempb{\linewidth}{\wd\pandoc@box}%
  \ifdim\@tempb\p@<\@tempa\p@\let\@tempa\@tempb\fi% select the smaller of both
  \ifdim\@tempa\p@<\p@\scalebox{\@tempa}{\usebox\pandoc@box}%
  \else\usebox{\pandoc@box}%
  \fi%
}
% Set default figure placement to htbp
\def\fps@figure{htbp}
\makeatother


% definitions for citeproc citations
\NewDocumentCommand\citeproctext{}{}
\NewDocumentCommand\citeproc{mm}{%
  \begingroup\def\citeproctext{#2}\cite{#1}\endgroup}
\makeatletter
 % allow citations to break across lines
 \let\@cite@ofmt\@firstofone
 % avoid brackets around text for \cite:
 \def\@biblabel#1{}
 \def\@cite#1#2{{#1\if@tempswa , #2\fi}}
\makeatother
\newlength{\cslhangindent}
\setlength{\cslhangindent}{1.5em}
\newlength{\csllabelwidth}
\setlength{\csllabelwidth}{3em}
\newenvironment{CSLReferences}[2] % #1 hanging-indent, #2 entry-spacing
 {\begin{list}{}{%
  \setlength{\itemindent}{0pt}
  \setlength{\leftmargin}{0pt}
  \setlength{\parsep}{0pt}
  % turn on hanging indent if param 1 is 1
  \ifodd #1
   \setlength{\leftmargin}{\cslhangindent}
   \setlength{\itemindent}{-1\cslhangindent}
  \fi
  % set entry spacing
  \setlength{\itemsep}{#2\baselineskip}}}
 {\end{list}}
\usepackage{calc}
\newcommand{\CSLBlock}[1]{\hfill\break\parbox[t]{\linewidth}{\strut\ignorespaces#1\strut}}
\newcommand{\CSLLeftMargin}[1]{\parbox[t]{\csllabelwidth}{\strut#1\strut}}
\newcommand{\CSLRightInline}[1]{\parbox[t]{\linewidth - \csllabelwidth}{\strut#1\strut}}
\newcommand{\CSLIndent}[1]{\hspace{\cslhangindent}#1}



\setlength{\emergencystretch}{3em} % prevent overfull lines

\providecommand{\tightlist}{%
  \setlength{\itemsep}{0pt}\setlength{\parskip}{0pt}}



 


\makeatletter
\@ifpackageloaded{caption}{}{\usepackage{caption}}
\AtBeginDocument{%
\ifdefined\contentsname
  \renewcommand*\contentsname{Table of contents}
\else
  \newcommand\contentsname{Table of contents}
\fi
\ifdefined\listfigurename
  \renewcommand*\listfigurename{List of Figures}
\else
  \newcommand\listfigurename{List of Figures}
\fi
\ifdefined\listtablename
  \renewcommand*\listtablename{List of Tables}
\else
  \newcommand\listtablename{List of Tables}
\fi
\ifdefined\figurename
  \renewcommand*\figurename{Figure}
\else
  \newcommand\figurename{Figure}
\fi
\ifdefined\tablename
  \renewcommand*\tablename{Table}
\else
  \newcommand\tablename{Table}
\fi
}
\@ifpackageloaded{float}{}{\usepackage{float}}
\floatstyle{ruled}
\@ifundefined{c@chapter}{\newfloat{codelisting}{h}{lop}}{\newfloat{codelisting}{h}{lop}[chapter]}
\floatname{codelisting}{Listing}
\newcommand*\listoflistings{\listof{codelisting}{List of Listings}}
\makeatother
\makeatletter
\makeatother
\makeatletter
\@ifpackageloaded{caption}{}{\usepackage{caption}}
\@ifpackageloaded{subcaption}{}{\usepackage{subcaption}}
\makeatother
\usepackage{bookmark}
\IfFileExists{xurl.sty}{\usepackage{xurl}}{} % add URL line breaks if available
\urlstyle{same}
\hypersetup{
  pdftitle={Reconstructing food webs in deep time: why model choice matters for ecological inference},
  pdfauthor={Tanya Strydom; Baran Karapunar; Andrew P. Beckerman; Alexander Dunhill},
  pdfkeywords={Paleoecological networks, Food-web
reconstruction, Ecological networks, Extinction dynamics, Trophic
interactions, Toarcian Oceanic Anoxic Event},
  colorlinks=true,
  linkcolor={blue},
  filecolor={Maroon},
  citecolor={Blue},
  urlcolor={Blue},
  pdfcreator={LaTeX via pandoc}}



\title{Reconstructing food webs in deep time: why model choice matters
for ecological inference}
\author{Tanya Strydom %
%
\textsuperscript{%
%
1%
}%
; Baran Karapunar %
%
\textsuperscript{%
%
2%
}%
; Andrew P. Beckerman %
%
\textsuperscript{%
%
1%
}%
; Alexander Dunhill %
%
\textsuperscript{%
%
2%
}%
}
\date{2026-02-04}

\usepackage{setspace}
\usepackage[left]{lineno}
\usepackage[letterpaper]{geometry}

\usepackage[nolists,noheads,markers]{endfloat}
\geometry{margin=2.5cm}

\begin{document}

\thispagestyle{empty}
{\bfseries\sffamily\Large Reconstructing food webs in deep time: why
model choice matters for ecological inference}
\vfil
Tanya Strydom %
%
\textsuperscript{%
%
1%
}%
; Baran Karapunar %
%
\textsuperscript{%
%
2%
}%
; Andrew P. Beckerman %
%
\textsuperscript{%
%
1%
}%
; Alexander Dunhill %
%
\textsuperscript{%
%
2%
}%

\vfil
{\small
\textbf{Abstract:} Food webs provide a powerful framework for
understanding ecosystem structure and function, yet reconstructing them
in paleoecological contexts remains challenging because direct evidence
of species interactions is rarely preserved. A growing array of models
exists for predicting interactions and inferring network structure, but
these approaches differ markedly in their assumptions, mechanisms, and
data requirements. Here, we evaluate how network reconstruction model
choice shapes ecological inference in deep time and which approaches are
most appropriate given the constraints of the fossil record. Using the
Toarcian Oceanic Anoxic Event (Early Jurassic, \textasciitilde183 Ma) as
a case study, we reconstruct food webs for four successive community
states using six contrasting modelling approaches spanning mechanistic,
trait-based, and structural network representations. Despite identical
taxon pools, models produced strikingly different network structures and
species-level interactions, and these differences propagated into
divergent inferences about extinction dynamics, interaction loss, and
cascading effects. By framing food-web reconstructions as alternative
ecological hypotheses (rather than interchangeable representations) this
study bridges paleoecology and modern network theory, and demonstrates
how model choice fundamentally conditions inference about extinction
dynamics in deep time.
\vfil
\textbf{Keywords:} %
Paleoecological networks, Food-web reconstruction, Ecological
networks, Extinction dynamics, Trophic interactions, %
Toarcian Oceanic Anoxic Event%
}
\clearpage
\setcounter{page}{1}
\doublespacing
\linenumbers


\section{Introduction}\label{introduction}

There is growing interest in using deep-time fossil data and evidence of
species interactions in past ecosystems (\emph{e.g.,} Dunne et al.
(2008); Dunne et al. (2014)) as a foundation for understanding how
ecological communities respond to multi-stressor environmental change,
such as temperature increase, ocean acidification, and hypoxia (Dillon
et al., 2022; Kiessling et al., 2019). Paleoecological networks
therefore represent a particularly valuable opportunity to investigate
community responses to major environmental transitions as they allow for
the explicit construction of pre- and post-extinction interaction
networks and enable the disentangling of extinction drivers as well as
broader cascading effects (Dunhill et al., 2024; Roopnarine, 2006;
Yeakel et al., 2014). Insights gained from these deep-time systems
provide a critical context for interpreting present day ecological
change and anticipating the impacts of ongoing anthropogenic stressors
(Barnosky et al., 2012; Roopnarine \& Dineen, 2018).

Inference from ecological networks regarding structure and complexity is
(at its core) a mathematical task and is therefore relatively
independent of biological assumptions (Delmas et al., 2019). While these
structural properties can be analysed independently of biological
detail, their implications for stability and dynamics depend critically
on assumptions about the distribution and strength of interaction
(Allesina \& Tang, 2012; Poisot et al., 2015). Thus, assumptions become
meaningful once network structure and complexity are interpreted in
functional terms or used as templates for simulating community and
extinction dynamics. While there is a well-developed theoretical
framework describing what can be inferred from network properties, a
central challenge (particularly in paleoecology) lies in how networks
are constructed. Unlike modern systems, paleoecological interactions are
rarely directly observed, with only exceptional cases preserving
explicit evidence of trophic interactions (\emph{e.g.,} Jenny et al.
(2019); Vullo (2011)). As a result, paleo food web reconstruction
depends on indirect inference, drawing on preserved traits, analogies
with modern taxa, and ecological theory. Simply put, network
reconstruction is not a data recovery process, but rather hypothesis
generation under constraints.

Although numerous tools exist for inferring species interactions (see
Morales-Castilla et al., 2015; Pichler \& Hartig, 2023; Strydom et al.,
2021; Allesina et al., 2008 for reviews), only a subset can be reliably
applied in paleo contexts, where data on traits, abundances, and
community composition are incomplete and systematically biased. This
makes it essential to clearly articulate which reconstruction approaches
are appropriate for which inferential purposes. Recent work has shown
that reconstruction approaches (\emph{e.g.,} models based on traits,
abundances, or co-occurrence) can be categorised by the type of network
that they predict (Strydom et al., 2026). These include feasible
networks (derived from trait matching and phylogenetic inference that
produce metawebs of plausible interactions), realised networks,
(constrained by ecological rules and evidence of actual feeding
relationships), and purely structural networks (which reproduce
ecologically plausible topologies but lack species-level node
identities). In this sense, food-web reconstructions are not empirical
recoveries of a single past ecosystem, but rather represent alternative,
model-based, hypotheses about interaction structure constrained by the
fossil record.

Each of these reconstruction approaches carries distinct assumptions
that influence inferred network size, complexity, structure, and
node-level properties, with direct consequences for ecological
interpretations of extinction dynamics, stability, resilience, and
ecosystem function (Dunne et al., 2002; Gravel et al., 2019; Roopnarine,
2006; Solé \& Montoya, 2001). Despite this, most paleo food web studies
default to constructing networks using expert knowledge (\emph{e.g.,}
Dunne et al. (2008)) or mechanistic feeding rules (\emph{e.g.,} Dunhill
et al. (2024); Roopnarine (2017); Fricke et al. (2022)), approaches that
typically result in metawebs. This raises a critical but underexplored
question: to what extent does the choice of network type (and
particularly the use of alternatives to metawebs) control not only
inferred food web structure but also conclusions about system behaviour,
especially with respect to extinction cascades and vulnerability.

In this study, we address this question by explicitly evaluating how
network reconstruction model choice shapes ecological inference in deep
time. We link recent advances in food-web reconstruction methods to a
comparative re-evaluation of primary and secondary, trait-based
extinction dynamics during the early Toarcian extinction event, a
volcanic-driven hyperthermal and marine crisis in the Early Jurassic
(\textasciitilde183 Ma) (Dunhill et al., 2024). We reconstruct four
successive communities (pre-extinction, post-extinction, early recovery,
and late recovery) from the Cleveland Basin of North Yorkshire, UK,
using six contrasting network reconstruction approaches spanning
feasible, realised, and structural network representations - as
recognised in recent network inference frameworks (Morales-Castilla et
al., 2015; Strydom et al., 2026). For each community, we compare
network-level structure, species-level properties, and inferred
interactions across models, allowing us to assess how reconstruction
assumptions propagate into ecological interpretations. Finally, we
replicate the secondary extinction analyses of Dunhill et al.~across all
six reconstruction methods to explicitly test how model choice
influences inference about extinction drivers, interaction loss, and
cascading dynamics. By explicitly comparing multiple reconstruction
approaches within a single paleoecological system, this study provides a
framework for evaluating how methodological assumptions shape
interpretations of ancient food-web structure and dynamics.

\section{Materials and Methods}\label{materials-and-methods}

\subsection{Study system and fossil
data}\label{study-system-and-fossil-data}

We used fossil occurrence data from the Cleveland Basin spanning the
upper Pliensbachian to the upper Toarcian, following Dunhill et al.
(2024). Four paleo-communities were defined: pre-extinction,
post-extinction, early recovery, and late recovery. Each assemblage was
treated as a community of potentially interacting taxa. Modes of life
were assigned following Dunhill et al. (2024) using the Bambach ecospace
framework. Traits included motility, tiering, feeding mode, and size
class, with size defined categorically based on maximum linear
dimensions.

\subsection{Network reconstruction
approaches}\label{network-reconstruction-approaches}

\subsubsection{Conceptual classification of network
types}\label{conceptual-classification-of-network-types}

Most paleo-specific approaches currently operate within the feasibility
space (\emph{e.g.,} Shaw et al., 2024; Fricke et al., 2022; Roopnarine,
2006). Although well suited for reconstructing feasible interactions,
these methods represent only a subset of the broader spectrum of network
construction approaches. Here, we present a suite of models
(Table~\ref{tbl-models}) that enable the construction of a wider range
of ecological networks and the exploration of a broader set of
ecological questions, provided that their underlying assumptions are
compatible with the constraints of fossil data. For example, some tools
require quantitative estimates of body size, which must often be
inferred from size classes or functional morphology in the fossil
record. Structural models, such as the niche model, rely only on species
richness and estimates/specification of connectance, but their
species-agnostic nature limits their applicability to trait-based or
diet-specific questions, although they do still accurately recover
network structure (Stouffer et al., 2005) . Mechanistic approaches, by
contrast, depend on accurate assignment of feeding traits or robust
phylogenetic support. Recognizing how these methodological requirements
intersect with the limits of fossil evidence is essential for selecting
an appropriate modelling framework.

\begin{longtable}[]{@{}
  >{\raggedright\arraybackslash}p{(\linewidth - 12\tabcolsep) * \real{0.1429}}
  >{\raggedright\arraybackslash}p{(\linewidth - 12\tabcolsep) * \real{0.1429}}
  >{\raggedright\arraybackslash}p{(\linewidth - 12\tabcolsep) * \real{0.1429}}
  >{\raggedright\arraybackslash}p{(\linewidth - 12\tabcolsep) * \real{0.1429}}
  >{\raggedright\arraybackslash}p{(\linewidth - 12\tabcolsep) * \real{0.1429}}
  >{\raggedright\arraybackslash}p{(\linewidth - 12\tabcolsep) * \real{0.1429}}
  >{\raggedright\arraybackslash}p{(\linewidth - 12\tabcolsep) * \real{0.1429}}@{}}
\caption{Six different models that can be used to construct food webs
for both this specific community but are also broadly suited to paleo
network prediction. These models span all facets of the network
representation space (metaweb, realised, and structural network) and are
suitable for an array of different paleo communities as the data
requirements fall within the limitations set by the fossil
record.}\label{tbl-models}\tabularnewline
\toprule\noalign{}
\begin{minipage}[b]{\linewidth}\raggedright
Model family
\end{minipage} & \begin{minipage}[b]{\linewidth}\raggedright
Assumptions
\end{minipage} & \begin{minipage}[b]{\linewidth}\raggedright
Data needs
\end{minipage} & \begin{minipage}[b]{\linewidth}\raggedright
`Limitation'
\end{minipage} & \begin{minipage}[b]{\linewidth}\raggedright
Network type
\end{minipage} & \begin{minipage}[b]{\linewidth}\raggedright
Key reference
\end{minipage} & \begin{minipage}[b]{\linewidth}\raggedright
Usage examples
\end{minipage} \\
\midrule\noalign{}
\endfirsthead
\toprule\noalign{}
\begin{minipage}[b]{\linewidth}\raggedright
Model family
\end{minipage} & \begin{minipage}[b]{\linewidth}\raggedright
Assumptions
\end{minipage} & \begin{minipage}[b]{\linewidth}\raggedright
Data needs
\end{minipage} & \begin{minipage}[b]{\linewidth}\raggedright
`Limitation'
\end{minipage} & \begin{minipage}[b]{\linewidth}\raggedright
Network type
\end{minipage} & \begin{minipage}[b]{\linewidth}\raggedright
Key reference
\end{minipage} & \begin{minipage}[b]{\linewidth}\raggedright
Usage examples
\end{minipage} \\
\midrule\noalign{}
\endhead
\bottomrule\noalign{}
\endlastfoot
random & Links are randomly distributed within a network & richness,
number of links & parameter assumptions, species agnostic & structural
network & Erdős \& Rényi (1959) & \\
niche & Networks are interval, species can be ordered on a `niche axis'
& richness, connectance & parameter assumptions, species agnostic &
structural network & Williams \& Martinez (2008) & \\
allometric diet breadth model (ADBM) & Interactions are determined by
energetic costs (foraging ecology) & body mass, biomass (abundance) &
does not account for forbidden links in terms of trait compatibility,
assumptions on body size and biomass (abundance) from fossil data &
realised network & Petchey et al. (2008) & \\
Allometric trophic network (ATN) & Interactions inferred using
allometric rules (ratio of body sizes between predator and prey), with
links being constrained by a Ricker function & body mass, number of
producer species & does not account for forbidden links in terms of
trait compatibility, assumptions on body size from fossil data,
assumptions as to the number of producer species & realised network &
Brose et al. (2006); Gauzens et al. (2023) & \\
paleo food web inference model (PFIM) & Interactions can be inferred by
a mechanistic framework/relationships & feeding traits for taxa,
mechanistic feeding rules & Assumption made as to the feeding
mechanisms, need to elucidate traits from models (although this is a way
smaller issue) & feasibility web & Shaw et al. (2024) & Secondary
extinctions (Dunhill et al., 2024) \\
body size ratio model & Interactions inferred using allometric rules
(ratio of body sizes between predator and prey). Logit of the linking
probability used to further constrain links to an `optimal size range'
for prey. & body mass & does not account for forbidden links in terms of
evolutionary compatibility, assumptions on body size from fossil data &
realised network & Rohr et al. (2010) & Network collapse (Yeakel et al.,
2014) \\
\end{longtable}

The three body-mass--based models (ADBM, ATN, body size ratio) differ
primarily in their underlying ecological assumptions. Although all three
models use body mass to infer food web structure, they differ in their
ecological assumptions. The ADBM is based on energy maximization under
optimal foraging theory, the ATN constrains interactions via
mechanically optimal consumer--resource size ratios, and the body size
ratio model defines links probabilistically within a fixed allometric
niche. Together, these approaches span bioenergetic, mechanical, and
statistical interpretations of size-structured interactions.

\subsubsection{Network generation and
replication}\label{network-generation-and-replication}

We evaluated six models spanning this space Table~\ref{tbl-models}:
random and niche models (structural network); allometric diet breadth
(ADBM), allometric trophic network (ATN), and body-size ratio models
(realised network); and a paleo food-web inference model (PFIM;
feasibility web). Expanded descriptions of model assumptions,
parameterisation, and link-generation rules are provided in
Supplementary Material S1. For each of the four communities, we
constructed 100 replicate networks using each of the six models (2400
networks total). Networks were simplified by removing disconnected
species. For size-based models, body masses were drawn from uniform
distributions bounded by size-class limits,allowing for variance between
replicates but preserving relative sizes within replicates. Structural
models were parameterised using connectance values drawn from an
empirically realistic range (0.07 -- 0.34) while holding richness
constant. The same parameter draws were used across comparable models
within each replicate.

\subsection{Network metrics and structural
analyses}\label{network-metrics-and-structural-analyses}

We quantified network structure using a suite of macro-, meso-, and
micro-scale metrics Table~\ref{tbl-properties}, capturing global
properties, motif structure, and species-level variability. Differences
among models were assessed using MANOVA, followed by univariate ANOVAs,
post-hoc comparisons, and linear discriminant analysis. Pairwise
interaction turnover was quantified using link‑based beta diversity,
which measures dissimilarity in the identity of trophic links between
networks, capturing differences due to species turnover or changes in
interactions among shared species (Poisot et al., 2012).

\begin{longtable}[]{@{}
  >{\raggedright\arraybackslash}p{(\linewidth - 6\tabcolsep) * \real{0.2466}}
  >{\raggedright\arraybackslash}p{(\linewidth - 6\tabcolsep) * \real{0.2603}}
  >{\raggedright\arraybackslash}p{(\linewidth - 6\tabcolsep) * \real{0.2466}}
  >{\raggedright\arraybackslash}p{(\linewidth - 6\tabcolsep) * \real{0.2466}}@{}}
\caption{Network properties used for
analysis.}\label{tbl-properties}\tabularnewline
\toprule\noalign{}
\begin{minipage}[b]{\linewidth}\raggedright
Metric
\end{minipage} & \begin{minipage}[b]{\linewidth}\raggedright
Definition
\end{minipage} & \begin{minipage}[b]{\linewidth}\raggedright
Scale
\end{minipage} & \begin{minipage}[b]{\linewidth}\raggedright
Reference (for maths), can make footnotes probs
\end{minipage} \\
\midrule\noalign{}
\endfirsthead
\toprule\noalign{}
\begin{minipage}[b]{\linewidth}\raggedright
Metric
\end{minipage} & \begin{minipage}[b]{\linewidth}\raggedright
Definition
\end{minipage} & \begin{minipage}[b]{\linewidth}\raggedright
Scale
\end{minipage} & \begin{minipage}[b]{\linewidth}\raggedright
Reference (for maths), can make footnotes probs
\end{minipage} \\
\midrule\noalign{}
\endhead
\bottomrule\noalign{}
\endlastfoot
Richness & Number of nodes in the network & Macro & \\
Links & Normalized standard deviation of links (number of consumers plus
resources per taxon) & Micro & \\
Connectance & \(L/S^2\), where \(S\) is the number of species and \(L\)
the number of links & Macro & \\
Max trophic level & Prey-weighted trophic level averaged across taxa &
Macro & Williams \& Martinez (2004) \\
S1 & Number of linear chains, normalised & Meso & Milo et al. (2002);
Stouffer et al. (2007) \\
S2 & Number of omnivory motifs, normalised & Meso & Milo et al. (2002);
Stouffer et al. (2007) \\
S4 & Number of apparent competition motifs, normalised & Meso & Milo et
al. (2002); Stouffer et al. (2007) \\
S5 & Number of direct competition motifs, normalised & Meso & Milo et
al. (2002); Stouffer et al. (2007) \\
Generality & Normalized standard deviation of generality of a species
standardized by \(L/S\) & Micro & Williams \& Martinez (2000) \\
Vulnerability & Normalized standard deviation of vulnerability of a
species standardized by \(L/S\) & Micro & Williams \& Martinez (2000) \\
\end{longtable}

\subsection{Extinction simulations and model
evaluation}\label{extinction-simulations-and-model-evaluation}

Following Dunhill et al. (2024) and using the pre-extinction and
post-extinction networks, we simulated species loss under multiple
extinction scenarios, including trait-based, network-position-based, and
random removals, allowing for secondary extinctions. Simulated
post-extinction networks were compared to empirical post-extinction
communities using mean absolute differences (MAD) in network metrics and
a modified true skill statistic (TSS) at both node and link levels.
Scenario rankings were compared across models using Kendall's rank
correlation coefficient.

\section{Results}\label{results}

Across six reconstruction approaches, both global network structure and
species-level interactions differed substantially, with implications for
interpreting past extinction dynamics. Deterministic models (e.g., PFIM)
tended to produce more consistent network-level patterns and smoother
extinction trajectories, whereas stochastic or theory-driven models
(\emph{e.g.,} ADBM, niche, ATN) showed greater variability in inferred
interactions and temporal extinction dynamics. Models with similar
macro-level metrics sometimes differed in their specification of
pairwise interactions, highlighting that agreement in global structure
does not guarantee concordance at the species level. Consequently,
inferred extinction pathways and secondary extinctions were highly
sensitive to model choice, emphasizing the importance of evaluating
multiple network reconstructions when interpreting ecological dynamics
in deep time.

\subsection{Network structure differs among reconstruction
approaches}\label{network-structure-differs-among-reconstruction-approaches}

To test whether network reconstruction approach influences inferred
food-web structure, we compared multivariate patterns of network metrics
across all six models using a MANOVA. Network structure differed
strongly among reconstruction approaches (MANOVA, Pillai's trace = 3.84,
approximate \(F_{40,11955}\) = 987.35, p \textless{} 0.001). Univariate
analyses showed that model choice explained a large proportion of
variance in most network metrics, with high partial η² values all
network structural metrics (η² = 0.65--0.92). Estimated marginal means
and Tukey-adjusted comparisons indicated consistent differences among
reconstruction approaches, with PFIM differing significantly from all
other models (\(p\) ≤ 0.0001). Within the allometric frameworks we
observed a notable divergence between the~ADBM~and~ATN models (\(p\) ≤
0.0001), demonstrating that bioenergetic ranking and
mechanical-efficiency rules do not converge on a single structural
solution. Interestingly, the only pair to exhibit statistical consensus
in multivariate space was the~ADBM~and the~log-ratio~model (\(p\) =
0.99). Linear discriminant analysis (LDA) further helped visualise
distinctions among reconstruction approaches in multivariate network
space Figure~\ref{fig-structure}, with the first two axes explaining
86\% of between-model variance (LD1 = 53\%, LD2 = 33\%). LD1 was most
strongly correlated with vulnerability (\(r\) = 0.86), direct
competition motifs (\(r\) = 0.81), and connectance (\(r\) = 0.75),
whereas LD2 was associated primarily with maximum trophic level (\(r\) =
−0.76) and a positive correlation with apparent competition motifs
(\(r\) = 0.73). Higher-order axes each explained less than 9\% of the
remaining variance. This demonstrates that the reconstruction approach
leaves a strong multivariate signature independent of community
composition.

\begin{figure}

\centering{

\pandocbounded{\includegraphics[keepaspectratio]{figures/MANOVA_lda.png}}

}

\caption{\label{fig-structure}Linear discriminant analysis (LDA) of
ecological network metrics for six model types. Each point represents a
replicate, and ellipses indicate 95\% confidence regions for each model.
The second column represents the correlation of the various network
metrics with the respective LDA axes.}

\end{figure}%

\subsubsection{Inferred pairwise interactions vary widely among
models}\label{inferred-pairwise-interactions-vary-widely-among-models}

Building on differences in global network structure, we next examined
how reconstruction approach influences species-level ecological
inference by quantifying turnover in inferred pairwise interactions
among networks constructed from the same taxon pool. While models that
produced similar global metrics sometimes agreed broadly on network
structure, they often differed sharply in the specific interactions they
inferred.

Pairwise β-turnover revealed that some model pairs shared very few links
despite comparable macro- or meso-scale properties
Figure~\ref{fig-beta_div}. ADBM and ATN were highly concordant,
reflecting similar underlying assumptions despite different generative
rules, whereas the body-size ratio model consistently exhibited high
differences in pairwise interactions relative to all other approaches.
PFIM showed intermediate overlap with size-based theoretical models.
These patterns indicate that agreement in global network metrics does
not guarantee agreement in species-level diets or trophic roles,
highlighting the importance of evaluating both network- and
species-level outcomes when comparing reconstruction methods.

\begin{figure}

\centering{

\pandocbounded{\includegraphics[keepaspectratio]{figures/beta_div.png}}

}

\caption{\label{fig-beta_div}Pairwise beta turnover in species
interactions among four ecological network models (ADBM, lmatrix,
body-size ratio, and pfim). Each cell represents the mean turnover value
between a pair of models, with warmer colors indicating greater
dissimilarity in inferred interactions. The diagonal is omitted. High
turnover values (yellow) indicate strong disagreement in network
structure between models, whereas lower values (blue--purple) indicate
greater similarity.}

\end{figure}%

\subsection{Model choice influences inferred extinction
dynamics}\label{model-choice-influences-inferred-extinction-dynamics}

To quantify how network structure changed over time during extinction
simulations and whether these dynamics differed among reconstruction
models, we fit generalized additive models (GAMs) to time series of
network-level metrics. GAMs capture nonlinear temporal trajectories,
allowing formal tests of whether the shape of these trajectories differs
among models. These model-specific temporal trajectories are shown in
Figure~\ref{fig-gam}. For all metrics examined, the inclusion of
model-specific smooth terms significantly improved model fit (ANOVA
model comparison: \(p\) \textless{} 0.001 for all metrics).
Deterministic, data-driven approaches (PFIM) and allometric models
(ADBM, ATN) exhibited highly non-linear trajectories, showing structural
shifts in connectivity and motif frequency. In contrast, the Niche model
produced the most consistent and gradual trajectories, effectively
smoothing the perceived magnitude of structural change during community
collapse. These results demonstrate that inferred pathways of collapse,
trophic bottlenecks, and secondary extinctions are highly sensitive to
model choice. Corresponding raw temporal trajectories are shown in Fig.
S3.

\begin{figure}

\centering{

\pandocbounded{\includegraphics[keepaspectratio]{figures/GAM_predictions.png}}

}

\caption{\label{fig-gam}GAM-predicted trajectories of network structure
during extinction simulations reveal pronounced differences in the
timing and magnitude of change across reconstruction models. Lines show
model-specific smooths and shaded areas indicate 95\% confidence
intervals. Deterministic approaches produce smoother, more consistent
dynamics, whereas stochastic models exhibit greater variability,
underscoring the sensitivity of inferred collapse pathways to
reconstruction assumptions.}

\end{figure}%

To evaluate how model choice affects inferred extinction dynamics, we
compared simulated post-extinction networks to observed networks using
mean absolute differences (MAD) for network-level metrics and total
sum-of-squares (TSS) for node- and link-level outcomes
Figure~\ref{fig-mad}. Overall, models were more consistent in ranking
extinction scenarios at the network level: Kendall's \(\tau\) values for
MAD-based rankings were generally positive, with strong agreement
between ADBM and ATN models (\(\tau\) ≈ 0.63). Node-level TSS scores
similarly showed broad consistency across models (\(\tau\): 0.25 --
0.90), reflecting comparable species removal sequences. In contrast,
link-level outcomes were more variable (\(\tau\): -0.48 - 0.29),
reflecting variance in the recovery of specific pairwise links between
real and simualted networks. These results indicate that while different
models often recover similar species-level extinction patterns, inferred
interaction loss and cascade dynamics are highly sensitive to model
choice.

\begin{figure}

\centering{

\pandocbounded{\includegraphics[keepaspectratio]{figures/kendal_tau.png}}

}

\caption{\label{fig-mad}Heatmaps showing pairwise Kendall rank
correlation coefficients (\(\tau\)) between models for each network
metric. Each panel corresponds to a different metric and displays the
degree of agreement in extinction-scenario rankings across models based
on mean absolute differences (MAD) between observed and predicted
network values. Positive \(\tau\) values (blue) indicate concordant
rankings between models, whereas negative \(\tau\) values (red) indicate
opposing rankings. Warmer colours approaching zero represent little or
no agreement. Panels illustrate how consistently different modelling
approaches evaluate the relative realism of extinction scenarios across
multiple network properties.}

\end{figure}%

\section{Discussion}\label{discussion}

\section{Model choice as a component of ecological
inference}\label{model-choice-as-a-component-of-ecological-inference}

Reconstructing food webs from fossil data is inherently an exercise in
inference under uncertainty. It involves not only assembling data but
also making explicit assumptions about how species interact and how
those interactions are represented mathematically (Dunne et al., 2008;
Morales-Castilla et al., 2015; Strydom et al., 2026). This process has
parallels in modern ecological network studies, where the tension
between data limitations and the goal of meaningful ecological inference
is well recognised (Delmas et al., 2019; Poisot et al., 2021). Results
demonstrate that the choice of network reconstruction model is itself a
major ecological decision, shaping not only the structural properties of
inferred networks but also downstream interpretations of extinction
dynamics (Allesina \& Tang, 2012; Solé \& Montoya, 2001).

These differences arise not from the fossil evidence \emph{per se}, but
from the assumptions embedded in each model family (Pichler \& Hartig,
2023; Strydom et al., 2021), such as how trophic links are defined
(trait compatibility versus energetic constraints), how interaction
probability is parameterised, and whether network topology is informed
by macroecological theory (\emph{e.g.,} niche structure) or by
mechanistic rules (\emph{e.g.,} body-size ratios). Consequently, network
reconstruction is not a neutral methodological step; model choice shapes
the ecological narratives we extract from ancient ecosystems. This
sensitivity mirrors challenges faced in modern network ecology, where
the choice of model and metric influences the interpretation of patterns
such as connectance, modularity, or motif distributions (Michalska-Smith
\& Allesina, 2019; Poisot \& Gravel, 2014) .

While previous studies have emphasized the role of model assumptions in
metaweb reconstruction (Dunhill et al., 2024; Roopnarine, 2006), our
results demonstrate that these assumptions create distinct, predictable
clusters of network properties. These clusters map directly onto the
conceptual divide between feasible, realised, and structural network
types (Strydom et al., 2026). Specifically, mechanistic models (PFIM)
identify a broad landscape of trait-compatible interactions, theoretical
models (ADBM, ATN, body size ratio) impose energetic filters to
approximate realised energy flow, and structural models (niche, random)
prioritise topological patterns over species identity.

Pairwise β-turnover analysis underscores that disagreements among
reconstruction approaches are not merely quantitative differences in
metrics, but qualitative differences in the identity of inferred
interactions. Models that may produce similar aggregate properties
(\emph{e.g.,} connectance) can still disagree strongly on species-level
diets and trophic roles. This reinforces concerns raised in both
paleoecological and modern studies that metrics alone can mask
substantive differences in network structure and function (Fricke et
al., 2022; Shaw et al., 2024).

The implications of these differences are most pronounced when
interpreting extinction dynamics (Dunne et al., 2002; Sahasrabudhe \&
Motter, 2011). While broad, trait-driven patterns of species loss are
relatively robust across models, the identity of lost interactions,
secondary extinctions, and cascade dynamics are sensitive to the type of
network reconstructed. Node-level patterns of species loss (such as
which taxa are more likely to go extinct under certain scenarios) tend
to be relatively robust across models, likely because they reflect
consistent trait-based vulnerabilities. However, inferred link-level
outcomes vary markedly with reconstruction assumptions as extinctions
are determined by network structure, \emph{i.e.,} are emergent
properties of model assumptions. This distinction mirrors findings in
modern food-web studies, where deterministic and stochastic model
assumptions influence the magnitude and timing of secondary extinctions
(Allesina \& Tang, 2012; Curtsdotter et al., 2011; Dunne et al., 2002;
Yeakel et al., 2014).

Taken together, these results highlight that network reconstruction is
not neutral. Rather, it is a hypothesis generation process where the
chosen model encodes a set of ecological assumptions. Consequently,
paleoecologists must carefully consider which ecological signals they
aim to recover (potential interactions, realised diets, or macro-scale
structural properties) before selecting a reconstruction approach.
Importantly, disagreement among models does not imply that any single
approach is `wrong', but rather reflects the fact that different models
capture different ecological signals (Stouffer, 2019). The challenge
therefore lies not in identifying a universally correct model, but
rather in aligning model choice with the ecological question being
asked. Recognising this is critical for advancing paleoecology beyond
descriptive reconstruction toward rigorous comparative inference.

\subsection{Aligning ecological questions with model
choice}\label{aligning-ecological-questions-with-model-choice}

A central insight from our study is that different ecological questions
require different network representations. This conclusion parallels
broader efforts in network ecology to clarify what various models and
metrics can validly infer about ecological systems (Gauzens et al.,
2025; Strydom et al., 2026). Here we provide a conceptual divide between
feasible, realised, and structural network types and provides a
practical framework for matching research goals with appropriate
reconstruction approaches.

\textbf{Feasibility networks:} (\emph{e.g.,} trait- and phylogeny-based
metaweb approaches) are best suited for questions about potential
trophic links and dietary breadth. These models aim to capture the range
of interactions that are biologically plausible given species traits,
even if not all are realised in any given context. Such an approach
aligns with metaweb concepts in modern ecology, where large pools of
potential interactions are used to understand regional species
interaction potentials and local assembly processes (Tylianakis \&
Morris, 2017).

\textbf{Realised networks:} (\emph{e.g.,} models incorporating energetic
and foraging constraints such as body-size allometry) are more
appropriate when the goal is to infer the most likely realised
interactions. These models embed ecological rules that approximate
energy transfer and foraging ecology, improving ecological plausibility
of predicted links as compared with purely combinatorial approaches
(Brose et al., 2006; Petchey et al., 2008).

\textbf{Structural networks:} (\emph{e.g.,} niche, cascade, and random
models) strip away species identities in favour of topological patterns,
and are useful when broad questions about connectance or trophic depth
are the focus. Structural models have a long history in network ecology
for generating null expectations about network topology (Allesina et
al., 2008; Williams \& Martinez, 2008).

Recognising this alignment helps avoid misinterpretation. For example,
reconstructing a metaweb and treating predicted links as realised
trophic interactions conflates potential with actual diet, potentially
exaggerating inferred interaction diversity.

\subsection{Implications for paleoecological network
studies}\label{implications-for-paleoecological-network-studies}

Our findings have three major implications for the field of
paleoecological networks:

\begin{enumerate}
\def\labelenumi{\arabic{enumi}.}
\item
  \textbf{Explicitly acknowledge model assumptions:} Interpretations of
  ancient food webs must clearly articulate the assumptions underlying
  reconstruction models. Without this, differences in networks
  reconstructed from different datasets or by different research groups
  may be misattributed to ecological differences rather than
  methodological choices.
\item
  \textbf{Standardise comparative frameworks:} When comparing food webs
  across studies, researchers should ensure that networks are
  constructed and analysed using comparable model families. Without such
  standardisation, meta-analyses risk conflating methodological
  differences with ecological or temporal variation.
\item
  \textbf{Leverage modern theory to expand inference:} Integrating
  modern network ecology frameworks and methods with paleo-specific
  approaches enriches the inferential toolkit available to
  paleoecologists (Dunne et al., 2014; Solé \& Montoya, 2001). Models
  developed for modern systems (\emph{e.g.,} allometric or trait-based
  energy models) can be adapted to the constraints of fossil data
  (\emph{e.g.,} Perez-Lamarque et al., 2026), enabling novel insights
  into deep-time dynamics.
\end{enumerate}

\subsection{Recommendations for network reconstruction in
paleoecology}\label{recommendations-for-network-reconstruction-in-paleoecology}

Given the sensitivity of ecological inference to reconstruction model
choice, we propose the following guidelines to improve consistency,
transparency, and ecological relevance:

\begin{enumerate}
\def\labelenumi{\arabic{enumi}.}
\item
  \textbf{Define the Inferential Goal First:}~Before reconstructing
  networks, researchers should articulate whether they aim to infer
  potential interactions, likely realised diets, or general structural
  properties. This will inform the selection of an appropriate model
  family consistent with the ecological question at hand (\emph{e.g.,}
  metaweb for complete diets, energetic models for trophic energy flows,
  or structural models for generic topologies)
\item
  \textbf{Use ensemble and sensitivity frameworks:} Rather than relying
  on a single model output, researchers should adopt ensemble approaches
  that generate and compare multiple network reconstructions. This not
  only quantifies model uncertainty but also reveals which ecological
  conclusions are robust biological signals and which are methodological
  artifacts.
\item
  \textbf{Standardise cross-study comparisons:} Comparisons of networks
  from different palaeoecological studies should be standardised by
  model family. When models differ, interpretations about ecological or
  environmental change should explicitly address how model choice may
  contribute to observed differences.
\item
  \textbf{Interpret scale-specific results with caution:} Because
  node-level patterns tend to be more robust to model choice than
  link-level patterns, researchers should prioritise interpretations at
  the appropriate scale. Structural conclusions about cascade pathways
  or secondary extinctions should be framed as model-dependent
  hypotheses rather than definitive historical reconstructions.
\end{enumerate}

\subsection{Future directions}\label{future-directions}

Looking ahead, paleoecological network reconstruction would benefit from
deeper integration with advances in modern network ecology. This
includes incorporating probabilistic and Bayesian approaches to quantify
uncertainty in link prediction, such as Bayesian group models
(Baskerville et al., 2011; Elmasri et al., 2020), developing maximum
entropy methods to predict network structure under constrained
information (Banville et al., 2023), and exploring multi-layer network
representations that integrate trophic interactions with other types of
ecological relationships (Pilosof et al., 2017). Such developments,
combined with increasing availability of trait and phylogenetic
information, can help bridge the gap between fossil constraints and
ecological inference, enabling more nuanced and probabilistically
grounded reconstructions of deep-time ecosystems (Banville et al., 2025;
Perez-Lamarque et al., 2026; Poisot et al., 2016).

\section{Conclusions}\label{conclusions}

Ecological network reconstruction in deep time is not merely a technical
step but a fundamental component of ecological inference. By explicitly
comparing six contrasting reconstruction approaches within a single
extinction event and location we show that model choice strongly shapes
inferred food-web structure, species interactions, and extinction
dynamics, even when underlying fossil data are identical. While broad,
trait-based patterns of species loss appear relatively robust,
conclusions about pairwise interactions, secondary extinctions, and
cascading dynamics depend critically on the assumptions embedded in the
chosen network reconstruction approach. These results underscore the
need for paleoecological studies to align reconstruction methods with
specific ecological questions and to evaluate the sensitivity of key
conclusions to alternative network representations. More broadly, our
findings highlight that understanding past ecosystem collapse requires
not only better fossil data, but also transparent, question-driven
modelling frameworks that make explicit the assumptions underlying
ecological inference.

\section*{References}\label{references}
\addcontentsline{toc}{section}{References}

\protect\phantomsection\label{refs}
\begin{CSLReferences}{1}{0}
\bibitem[\citeproctext]{ref-allesina2008}
Allesina, S., Alonso, D., \& Pascual, M. (2008). A general model for
food web structure. \emph{Science}, \emph{320}(5876), 658--661.
\url{https://doi.org/10.1126/science.1156269}

\bibitem[\citeproctext]{ref-allesina2012}
Allesina, S., \& Tang, S. (2012). Stability criteria for complex
ecosystems. \emph{Nature}, \emph{483}(7388), 205--208.
\url{https://doi.org/10.1038/nature10832}

\bibitem[\citeproctext]{ref-banville2023}
Banville, F., Gravel, D., \& Poisot, T. (2023). What constrains food
webs? A maximum entropy framework for predicting their structure with
minimal biases. \emph{PLOS Computational Biology}, \emph{19}(9),
e1011458. \url{https://doi.org/10.1371/journal.pcbi.1011458}

\bibitem[\citeproctext]{ref-banville2025}
Banville, F., Strydom, T., Blyth, P. S. A., Brimacombe, C., Catchen, M.
D., Dansereau, G., Higino, G., Malpas, T., Mayall, H., Norman, K.,
Gravel, D., \& Poisot, T. (2025). Deciphering probabilistic species
interaction networks. \emph{Ecology Letters}, \emph{28}(6), e70161.
\url{https://doi.org/10.1111/ele.70161}

\bibitem[\citeproctext]{ref-barnosky2012}
Barnosky, A. D., Hadly, E. A., Bascompte, J., Berlow, E. L., Brown, J.
H., Fortelius, M., Getz, W. M., Harte, J., Hastings, A., Marquet, P. A.,
Martinez, N. D., Mooers, A., Roopnarine, P., Vermeij, G., Williams, J.
W., Gillespie, R., Kitzes, J., Marshall, C., Matzke, N., \ldots{} Smith,
A. B. (2012). Approaching a state shift in earth{'}s biosphere.
\emph{Nature}, \emph{486}(7401), 52--58.
\url{https://doi.org/10.1038/nature11018}

\bibitem[\citeproctext]{ref-baskerville2011}
Baskerville, E. B., Dobson, A. P., Bedford, T., Allesina, S., Anderson,
T. M., \& Pascual, M. (2011). Spatial guilds in the serengeti food web
revealed by a bayesian group model. \emph{PLOS Computational Biology},
\emph{7}(12), e1002321.
\url{https://doi.org/10.1371/journal.pcbi.1002321}

\bibitem[\citeproctext]{ref-brose2006}
Brose, U., Jonsson, T., Berlow, E. L., Warren, P., Banasek-Richter, C.,
Bersier, L.-F., Blanchard, J. L., Brey, T., Carpenter, S. R.,
Blandenier, M.-F. C., Cushing, L., Dawah, H. A., Dell, T., Edwards, F.,
Harper-Smith, S., Jacob, U., Ledger, M. E., Martinez, N. D., Memmott,
J., \ldots{} Cohen, J. E. (2006). Consumer{\textendash}resource
body-size relationships in natural food webs. \emph{Ecology},
\emph{87}(10), 2411--2417.
https://doi.org/\url{https://doi.org/10.1890/0012-9658(2006)87\%5B2411:CBRINF\%5D2.0.CO;2}

\bibitem[\citeproctext]{ref-curtsdotter2011}
Curtsdotter, A., Binzer, A., Brose, U., De Castro, F., Ebenman, B.,
Eklöf, A., Riede, J. O., Thierry, A., \& Rall, B. C. (2011). Robustness
to secondary extinctions: Comparing trait-based sequential deletions in
static and dynamic food webs. \emph{Basic and Applied Ecology},
\emph{12}(7), 571--580. \url{https://doi.org/10.1016/j.baae.2011.09.008}

\bibitem[\citeproctext]{ref-delmas2019}
Delmas, E., Besson, M., Brice, M.-H., Burkle, L. A., Riva, G. V. D.,
Fortin, M.-J., Gravel, D., Guimarães, P. R., Hembry, D. H., Newman, E.
A., Olesen, J. M., Pires, M. M., Yeakel, J. D., \& Poisot, T. (2019).
Analysing ecological networks of species interactions. \emph{Biological
Reviews}, \emph{94}(1), 16--36. \url{https://doi.org/10.1111/brv.12433}

\bibitem[\citeproctext]{ref-dillon2022}
Dillon, E. M., Pier, J. Q., Smith, J. A., Raja, N. B., Dimitrijević, D.,
Austin, E. L., Cybulski, J. D., De Entrambasaguas, J., Durham, S. R.,
Grether, C. M., Haldar, H. S., Kocáková, K., Lin, C.-H., Mazzini, I.,
Mychajliw, A. M., Ollendorf, A. L., Pimiento, C., Regalado Fernández, O.
R., Smith, I. E., \& Dietl, G. P. (2022). What is conservation
paleobiology? Tracking 20 years of research and development.
\emph{Frontiers in Ecology and Evolution}, \emph{10}.
\url{https://doi.org/10.3389/fevo.2022.1031483}

\bibitem[\citeproctext]{ref-dunhill2024}
Dunhill, A. M., Zarzyczny, K., Shaw, J. O., Atkinson, J. W., Little, C.
T. S., \& Beckerman, A. P. (2024). Extinction cascades, community
collapse, and recovery across a mesozoic hyperthermal event.
\emph{Nature Communications}, \emph{15}(1), 8599.
\url{https://doi.org/10.1038/s41467-024-53000-2}

\bibitem[\citeproctext]{ref-dunne2014}
Dunne, J. A., Labandeira, C. C., \& Williams, R. J. (2014). Highly
resolved early eocene food webs show development of modern trophic
structure after the end-cretaceous extinction. \emph{Proceedings of the
Royal Society B: Biological Sciences}, \emph{281}(1782), 20133280.
\url{https://doi.org/10.1098/rspb.2013.3280}

\bibitem[\citeproctext]{ref-dunne2008}
Dunne, J. A., Williams, R. J., Martinez, N. D., Wood, R. A., \& Erwin,
D. H. (2008). Compilation and network analyses of cambrian food webs.
\emph{PLOS Biology}, \emph{6}(4), e102.
\url{https://doi.org/10.1371/journal.pbio.0060102}

\bibitem[\citeproctext]{ref-dunne2002}
Dunne, J., Williams, R. J., \& Martinez, N. D. (2002). Network structure
and biodiversity loss in food webs: Robustness increases with
connectance. \emph{Ecol. Lett.}, \emph{5}(4), 558--567.

\bibitem[\citeproctext]{ref-elmasri2020}
Elmasri, M., Farrell, M. J., Davies, T. J., \& Stephens, D. A. (2020). A
hierarchical bayesian model for predicting ecological interactions using
scaled evolutionary relationships. \emph{The Annals of Applied
Statistics}, \emph{14}(1), 221--240.
\url{https://doi.org/10.1214/19-AOAS1296}

\bibitem[\citeproctext]{ref-erdos1959}
Erdős, P., \& Rényi, A. (1959). On random graphs. i. \emph{Publicationes
Mathematicae Debrecen}, \emph{6}(3-4), 290--297.
\url{https://doi.org/10.5486/pmd.1959.6.3-4.12}

\bibitem[\citeproctext]{ref-fricke2022a}
Fricke, E. C., Hsieh, C., Middleton, O., Gorczynski, D., Cappello, C.
D., Sanisidro, O., Rowan, J., Svenning, J.-C., \& Beaudrot, L. (2022).
Collapse of terrestrial mammal food webs since the late pleistocene.
\emph{Science}, \emph{377}(6609), 1008--1011.
\url{https://doi.org/10.1126/science.abn4012}

\bibitem[\citeproctext]{ref-gauzens2023}
Gauzens, B., Brose, U., Delmas, E., \& Berti, E. (2023). ATNr:
Allometric trophic network models in r. \emph{Methods in Ecology and
Evolution}, \emph{14}(11), 2766--2773.
\url{https://doi.org/10.1111/2041-210X.14212}

\bibitem[\citeproctext]{ref-gauzens2025}
Gauzens, B., Thouvenot, L., Srivastava, D. S., Kratina, P., Romero, G.
Q., Berti, E., O'Gorman, E. J., González, A. L., Dézerald, O.,
Eisenhauer, N., Pires, M., Ryser, R., Farjalla, V. F., Rogy, P., Brose,
U., Petermann, J. S., Geslin, B., \& Hines, J. (2025). Tailoring
interaction network types to answer different ecological questions.
\emph{Nature Reviews Biodiversity}, 1--10.
\url{https://doi.org/10.1038/s44358-025-00056-7}

\bibitem[\citeproctext]{ref-gravel2019}
Gravel, D., Baiser, B., Dunne, J. A., Kopelke, J.-P., Martinez, N. D.,
Nyman, T., Poisot, T., Stouffer, D. B., Tylianakis, J. M., Wood, S. A.,
\& Roslin, T. (2019). Bringing elton and grinnell together: A
quantitative framework to represent the biogeography of ecological
interaction networks. \emph{Ecography}, \emph{42}(3), 401--415.
https://doi.org/\url{https://doi.org/10.1111/ecog.04006}

\bibitem[\citeproctext]{ref-jenny2019}
Jenny, D., Fuchs, D., Arkhipkin, A. I., Hauff, R. B., Fritschi, B., \&
Klug, C. (2019). Predatory behaviour and taphonomy of a jurassic
belemnoid coleoid (diplobelida, cephalopoda). \emph{Scientific Reports},
\emph{9}(1), 7944. \url{https://doi.org/10.1038/s41598-019-44260-w}

\bibitem[\citeproctext]{ref-kiessling2019}
Kiessling, W., Raja, N. B., Roden, V. J., Turvey, S. T., \& Saupe, E. E.
(2019). Addressing priority questions of conservation science with
palaeontological data. \emph{Philosophical Transactions of the Royal
Society B: Biological Sciences}, \emph{374}(1788), 20190222.
\url{https://doi.org/10.1098/rstb.2019.0222}

\bibitem[\citeproctext]{ref-michalska-smith2019}
Michalska-Smith, M. J., \& Allesina, S. (2019). Telling ecological
networks apart by their structure: A computational challenge. \emph{PLOS
Computational Biology}, \emph{15}(6), e1007076.
\url{https://doi.org/10.1371/journal.pcbi.1007076}

\bibitem[\citeproctext]{ref-milo2002}
Milo, R., Shen-Orr, S., Itzkovitz, S., Kashtan, N., Chklovskii, D., \&
Alon, U. (2002). Network motifs: Simple building blocks of complex
networks. \emph{Science}, \emph{298}(5594), 824--827.
\url{https://doi.org/10.1126/science.298.5594.824}

\bibitem[\citeproctext]{ref-morales-castilla2015}
Morales-Castilla, I., Matias, M. G., Gravel, D., \& Araújo, M. B.
(2015). Inferring biotic interactions from proxies. \emph{Trends in
Ecology \& Evolution}, \emph{30}(6), 347--356.
\url{https://doi.org/10.1016/j.tree.2015.03.014}

\bibitem[\citeproctext]{ref-perez-lamarque2026}
Perez-Lamarque, B., Andréoletti, J., Morillon, B., Pion-Piola, O.,
Lambert, A., \& Morlon, H. (2026). Darwin{'}s entangled bank through
deep time: Structural stability of mutualistic networks over large
geographic and temporal scales. \emph{EcoEvoRxiv}.
\url{https://doi.org/10.1101/2025.10.08.681159}

\bibitem[\citeproctext]{ref-petchey2008}
Petchey, O. L., Beckerman, A. P., Riede, J. O., \& Warren, P. H. (2008).
Size, foraging, and food web structure. \emph{Proceedings of the
National Academy of Sciences}, \emph{105}(11), 4191--4196.
\url{https://doi.org/10.1073/pnas.0710672105}

\bibitem[\citeproctext]{ref-pichler2023}
Pichler, M., \& Hartig, F. (2023). Machine learning and deep
learning{\textemdash}a review for ecologists. \emph{Methods in Ecology
and Evolution}, \emph{14}(4), 994--1016.
\url{https://doi.org/10.1111/2041-210X.14061}

\bibitem[\citeproctext]{ref-pilosof2017}
Pilosof, S., Porter, M. A., Pascual, M., \& Kéfi, S. (2017). The
multilayer nature of ecological networks. \emph{Nature Ecology \&
Evolution}, \emph{1}(4), 101.
\url{https://doi.org/10.1038/s41559-017-0101}

\bibitem[\citeproctext]{ref-poisot2021}
Poisot, T., Bergeron, G., Cazelles, K., Dallas, T., Gravel, D.,
MacDonald, A., Mercier, B., Violet, C., \& Vissault, S. (2021). Global
knowledge gaps in species interaction networks data. \emph{Journal of
Biogeography}, jbi.14127. \url{https://doi.org/10.1111/jbi.14127}

\bibitem[\citeproctext]{ref-poisot2012a}
Poisot, T., Canard, E., Mouillot, D., Mouquet, N., \& Gravel, D. (2012).
The dissimilarity of species interaction networks. \emph{Ecology
Letters}, \emph{15}(12), 1353--1361.
\url{https://doi.org/10.1111/ele.12002}

\bibitem[\citeproctext]{ref-poisot2016}
Poisot, T., Cirtwill, A., Cazelles, K., Gravel, D., Fortin, M.-J., \&
Stouffer, D. (2016). The structure of probabilistic networks.
\emph{Methods in Ecology and Evolution}, \emph{7}(3), 303312.
\url{https://doi.org/10}

\bibitem[\citeproctext]{ref-poisot2014}
Poisot, T., \& Gravel, D. (2014). When is an ecological network complex?
Connectance drives degree distribution and emerging network properties.
\emph{PeerJ}, \emph{2}, e251. \url{https://doi.org/10.7717/peerj.251}

\bibitem[\citeproctext]{ref-poisot2015}
Poisot, T., Stouffer, D. B., \& Gravel, D. (2015). Beyond species: Why
ecological interaction networks vary through space and time.
\emph{Oikos}, \emph{124}(3), 243--251.
\url{https://doi.org/10.1111/oik.01719}

\bibitem[\citeproctext]{ref-rohr2010}
Rohr, R., Scherer, H., Kehrli, P., Mazza, C., \& Bersier, L.-F. (2010).
Modeling food webs: Exploring unexplained structure using latent traits.
\emph{The American Naturalist}, \emph{176}(2), 170--177.
\url{https://doi.org/10.1086/653667}

\bibitem[\citeproctext]{ref-roopnarine2017}
Roopnarine, P. D. (2017). \emph{Ecological modelling of paleocommunity
food webs} (pp. 201--226). University of Chicago Press.

\bibitem[\citeproctext]{ref-roopnarine2006}
Roopnarine, P. D. (2006). Extinction cascades and catastrophe in ancient
food webs. \emph{Paleobiology}, \emph{32}(1), 1--19.
\url{https://www.jstor.org/stable/4096814}

\bibitem[\citeproctext]{ref-roopnarine2018}
Roopnarine, P. D., \& Dineen, A. A. (2018). \emph{Coral reefs in crisis:
The reliability of deep-time food web reconstructions as analogs for the
present} (C. L. Tyler \& C. L. Schneider, Eds.; pp. 105--141). Springer
International Publishing.
\url{https://doi.org/10.1007/978-3-319-73795-9_6}

\bibitem[\citeproctext]{ref-sahasrabudhe2011}
Sahasrabudhe, S., \& Motter, A. E. (2011). Rescuing ecosystems from
extinction cascades through compensatory perturbations. \emph{Nature
Communications}, \emph{2}(1), 170.
\url{https://doi.org/10.1038/ncomms1163}

\bibitem[\citeproctext]{ref-shaw2024}
Shaw, J. O., Dunhill, A. M., Beckerman, A. P., Dunne, J. A., \& Hull, P.
M. (2024). \emph{A framework for reconstructing ancient food webs using
functional trait data} (p. 2024.01.30.578036). bioRxiv.
\url{https://doi.org/10.1101/2024.01.30.578036}

\bibitem[\citeproctext]{ref-soluxe92001}
Solé, R. V., \& Montoya, M. (2001). Complexity and fragility in
ecological networks. \emph{Proceedings of the Royal Society of London.
Series B: Biological Sciences}, \emph{268}(1480), 2039--2045.
\url{https://doi.org/10.1098/rspb.2001.1767}

\bibitem[\citeproctext]{ref-stouffer2005}
Stouffer, D. B., Camacho, J., Guimerà, R., Ng, C. A., \& Nunes Amaral,
L. A. (2005). Quantitative patterns in the structure of model and
empirical food webs. \emph{Ecology}, \emph{86}(5), 1301--1311.
\url{https://doi.org/10.1890/04-0957}

\bibitem[\citeproctext]{ref-stouffer2019}
Stouffer, D. B. (2019). All ecological models are wrong, but some are
useful. \emph{Journal of Animal Ecology}, \emph{88}(2), 192--195.
https://doi.org/\url{https://doi.org/10.1111/1365-2656.12949}

\bibitem[\citeproctext]{ref-stouffer2007}
Stouffer, D. B., Camacho, J., Jiang, W., \& Nunes Amaral, L. A. (2007).
Evidence for the existence of a robust pattern of prey selection in food
webs. \emph{Proceedings of the Royal Society B: Biological Sciences},
\emph{274}(1621), 1931--1940.
\url{https://doi.org/10.1098/rspb.2007.0571}

\bibitem[\citeproctext]{ref-strydom2021}
Strydom, T., Catchen, M. D., Banville, F., Caron, D., Dansereau, G.,
Desjardins-Proulx, P., Forero-Muñoz, N. R., Higino, G., Mercier, B.,
Gonzalez, A., Gravel, D., Pollock, L., \& Poisot, T. (2021). A roadmap
towards predicting species interaction networks (across space and time).
\emph{Philosophical Transactions of the Royal Society B: Biological
Sciences}, \emph{376}(1837), 20210063.
\url{https://doi.org/10.1098/rstb.2021.0063}

\bibitem[\citeproctext]{ref-strydom2026}
Strydom, T., Dunhill, A. M., Dunne, J. A., Poisot, T., \& Beckerman, A.
P. (2026). Scaling from metawebs to realised webs: A hierarchical
approach to network ecology. \emph{EcoEvoRxiv}.
\url{https://doi.org/10.32942/X2JW8K}

\bibitem[\citeproctext]{ref-tylianakis2017}
Tylianakis, J. M., \& Morris, R. J. (2017). Ecological networks across
environmental gradients. \emph{Annual Review of Ecology, Evolution, and
Systematics}, \emph{48}(1), 25--48.
\url{https://doi.org/10.1146/annurev-ecolsys-110316-022821}

\bibitem[\citeproctext]{ref-vullo2011}
Vullo, R. (2011). Direct evidence of hybodont shark predation on late
jurassic ammonites. \emph{Naturwissenschaften}, \emph{98}(6), 545--549.
\url{https://doi.org/10.1007/s00114-011-0789-9}

\bibitem[\citeproctext]{ref-williams2004}
Williams, R. J., \& Martinez, N. D. (2004). Stabilization of chaotic and
non-permanent food-web dynamics. \emph{The European Physical Journal B -
Condensed Matter}, \emph{38}(2), 297--303.
\url{https://doi.org/10.1140/epjb/e2004-00122-1}

\bibitem[\citeproctext]{ref-williams2000}
Williams, R. J., \& Martinez, N. D. (2000). Simple rules yield complex
food webs. \emph{Nature}, \emph{404}(6774), 180--183.
\url{https://doi.org/10.1038/35004572}

\bibitem[\citeproctext]{ref-williams2008}
Williams, R. J., \& Martinez, N. D. (2008). Success and its limits among
structural models of complex food webs. \emph{The Journal of Animal
Ecology}, \emph{77}(3), 512--519.
\url{https://doi.org/10.1111/j.1365-2656.2008.01362.x}

\bibitem[\citeproctext]{ref-yeakel2014}
Yeakel, J. D., Pires, M. M., Rudolf, L., Dominy, N. J., Koch, P. L.,
Guimarães, P. R., \& Gross, T. (2014). Collapse of an ecological network
in ancient egypt. \emph{PNAS}, \emph{111}(40), 14472--14477.
\url{https://doi.org/10.1073/pnas.1408471111}

\end{CSLReferences}





\end{document}
